The propagation diagrams above illustrate some of the general similarity between various types of g mode pulsators. Although very different in terms of mass, evolutionary state, etc., each of these models contains a convective core surrounded by a radiative envelope that comprises the g mode cavity. An approximate rule of thumb is that field strengths of order $B \sim 10^5 \, {\rm G}$ just outside the convective core, or field strengths of order $B \sim 10^3 \, {\rm G}$ near the surface, are required for strong g mode alteration. The value of $B_c$ at the edge of the convective core should be interpreted cautiously because it depends on mixing/overshoot processes that may substantially alter the value of $N$ and therefore the value of $B_c$ at this location. Values of $B_c$ for different mode frequencies $\nu$ can be calculated using $B_c \propto \nu^2$. 