\subsection{Magnetic Field Structure}
\label{fieldstruc}

The spatial structure of a MS dynamo-generated field has important implications for the suppression of oscillation modes during the red giant phase.
%As discussed above, the radial extent of the field will affect the mass range over which the magnetic greenhouse effect is likely to occur.
The angular structure of the field will affect magnetic mode splitting and wave scattering, which could potentially be used to constrain the magnetic field structure within red giants.

While the dynamo is active, we expect the magnetic field to vary on horizontal scales comparable to those of convective eddies, as seen in simulations (\citealt{Featherstone_2009}, Augustson et al. 2015). The largest convective eddies at the outer edge of the convective core have length scales of $\sim \! H_P \sim \! r_c$, where $r_c$ is the radius of the convective core. Since these magnetic structures can be long-lived, we expect them to be mostly frozen in to the core when it becomes radiative at the end of the MS, as long as the global field structure remains stable. Therefore, we expect dominant fluctuations in the angular structure of the field to occur on length scales of $\sim \! r_c$, although smaller structures may also exist due to smaller eddies within the convective flow.

To extrapolate these structures into the red giant phase, we assume they remain frozen in place as the core contracts. During this process, the structures shrink by a factor of $r_{\rm RG}/r_{\rm MS}$, where $r_{\rm MS}$ is the radial coordinate of a mass shell on the MS and $r_{\rm RG}$ is its corresponding location in a red giant. Hence, we expect the field to vary on horizontal length scales up to the radial coordinate $r$ in red giant cores. The Ohmic diffusion time across structures of this size is large (see Section \ref{time}), such that they are not able to be significantly smoothed out within the lifetime of the star.

Horizontal variations break the spherical symmetry of the field and affect its interaction with oscillations. In principle, one could envision a field structure that is tangled only on very small scales $l$, such that its surface resembles the dimpled surface of a cantaloupe and appears nearly spherically symmetric to an incoming wave. For this to happen, the length scale $l$ must be smaller than the {\it radial} wavelength of incoming waves such that $k_r l < 1.$ However, we argued above that the field will vary on length scales $r$, and Fuller et al. 2015 showed that $k_r r \gg 1$ within the cores of red giants for observable oscillation modes. Therefore, realistic field configurations can always break the spherical symmetry of the background and scatter incoming waves into high angular wave numbers $k_\perp$ such that the magnetic greenhouse effect opearates as described in Fuller et al. 2015.