\section{Magnetic Mode Suppression in Red Clump Stars}
\label{clump}

The understanding of magnetic mode suppression presented in Section \ref{rgb} can be used to make predictions for magnetic mode suppression in red clump stars. We expect that mode suppression will occur if strong fields created by convective core dynamos can exist within the radiative regions of clump stars. 


Figures \ref{fig:ClumpProp} and \ref{fig:ClumpPropMassive} show propagation diagrams of clump stars with ZAMS masses of $1.5 \, M_\odot$ and $2.5 \, M_\odot$, respectively. 
%The stellar models have been computed using the MESA stellar evolution code \citep{Paxton:2011,Paxton:2013} using the inlist provided in Appendix~\ref{inlist}.
%Convective boundaries have been calculated using the Schwarzschild criterion as appropriate for stars of this mass \citep{Moore_2015}
Again, to calculate plausible magnetic field strengths in the cores of these stars, we assume that the MS convective core dynamo creates a magnetic field of equipartition strength.
%\begin{equation}
%\label{eqn:Beq}
%B_{\rm MS} = \sqrt{ 8 \pi \rho v_{\rm con}^2} \, .
%\end{equation}
%Here, $v_{\rm con}$ is the convective velocity according to MLT, $v_{\rm con} \approx (F/\rho)^{1/3}$, and $F$ is the energy flux carried by convection. 
Using Eq.~\ref{eqn:Beq} we find typical field strengths of $B_{\rm MS} \sim 2 \times 10^5 \, {\rm G}$; again due to the very long diffusion timescales, we expect these fields to be confined to the maximal extent of the core convective region. We then assume that magnetic flux is conserved as the star evolves into a red giant, such that
\begin{equation}
\label{eqn:Brgb}
B_{\rm RG} = B_{\rm MS} \bigg( \frac{r_{\rm MS}}{r_{\rm RG}} \bigg)^2 \, .
\end{equation}
Here, the red giant magnetic field $B_{\rm RG}$ at a mass coordinate $m$ is calculated from the corresponding MS field $B_{\rm MS}$ at the moment when the convective core has its largest extent. The radius $r_{\rm MS}$ is the radial coordinate of the mass shell at this point on the main sequence, while the radius $r_{\rm RG}$ is the radial coordinate of the mass shell when the star is on the clump. As shown in the bottom panel of Figures \ref{fig:ClumpProp} and \ref{fig:ClumpPropMassive}, magnetic fields of $B_{\rm RG} > 10^6 \, {\rm G}$ may exist in the cores of clump stars. 

Figures \ref{fig:ClumpProp} and \ref{fig:ClumpPropMassive} also show the magneto-gravity frequency $\omega_{\rm MG}$ \citep{Fuller_2015}
\begin{equation}
\label{eqn:omegaMG}
\omega_{\rm MG} = \bigg[ \frac{ 2 B_{\rm RG}^2 N^2}{\pi \rho r^2} \bigg]^{1/4} \, .
\end{equation}
Magnetic suppression is expected if $\nu_{\rm max} < \omega_{\rm MG}/(2 \pi)$ at some point in the radiative region. This is equivalent to the requirement that $B_{\rm RG} > B_c$, where $B_c$ is the critical magnetic field strength
\begin{equation}
\label{eqn:Bc}
B_c = \sqrt{ \frac{ \pi \rho}{2} } \frac{\omega^2 r}{N} \, ,
\end{equation}
evaluated at angular wave frequencies $\omega = 2 \pi \nu_{\rm max}$. Figure \ref{fig:BcClump} shows the minimum field strength required for mode suppression, $B_{c,{\rm min}}$, evaluated at the peak in $N$ at the H-burning shell, for stars on the red clump. In general, fields of $10^4$-$2 \! \times \! 10^5 \, {\rm G}$ are sufficient for mode suppression. 

Figure \ref{fig:ClumpProp} and \ref{fig:ClumpPropMassive} indicate that magnetic fields strong enough to cause mode suppression may commonly exist within the radiative cores of clump stars. In stars of $M \lesssim 2.25 \, M_\odot$, these fields will only exist below the H-burning shell, however, they are likely still strong enough to cause magnetic mode suppression. In stars of $M \gtrsim 2.25 \, M_\odot$, the strong fields extend beyond the H-burning shell and will almost certainly lead to mode suppression. These conclusions are not sensitive to the spike in $N$ just above the convective core (which occurs because of the composition gradient between convective core and radiative region) and are not sensitive to mixing processes at the convective core boundary. 

Inspection of Figures \ref{fig:DipoleHist} and \ref{fig:ClumpProp} demonstrates another interesting feature: in stars of $M \lesssim 1.4 M_\odot$, the mass of the convective core on the clump is larger than its maximal extent on the main sequence. Therefore, we do not expect strong fields in the radiative regions of low-mass $M \lesssim 1.4 M_\odot$ clump stars, and we predict that low-mass clump stars will rarely exhibit dipole mode suppression. Hence, mode suppression on the clump may be similar to that measured for RGB stars measured by Stello et al. 2015: mode suppression will be rare for stars with $M \lesssim 1.3 M_\odot$ (occurring in less than $\sim \! 10 \%$ of stars), and common for stars $M \gtrsim 1.5 M_\odot$ (occurring in greater than $\sim \! 50 \%$ of stars). 


    
  
  
  
  
  