\subsection{Magnetic Field Strength and Structure}
\label{fieldstruc}

To calculate plausible magnetic field strengths in the cores of red giant stars, we assume that the MS convective core dynamo creates a magnetic field of equipartition strength.
%\begin{equation}
%\label{eqn:Beq}
%B_{\rm MS} = \sqrt{ 8 \pi \rho v_{\rm con}^2} \, .
%\end{equation}
%Here, $v_{\rm con}$ is the convective velocity according to MLT, $v_{\rm con} \approx (F/\rho)^{1/3}$, and $F$ is the energy flux carried by convection. 
Using Eq.~\ref{eqn:Beq} we find typical field strengths of $B_{\rm MS} \sim 2 \! \times \! 10^5 \, {\rm G}$; again due to the very long diffusion timescales, we expect these fields to be confined to the maximal extent of the core convective region. As in \citet{Fuller_2015} we then assume that magnetic flux is conserved as the star evolves into a red giant, such that
\begin{equation}
\label{eqn:Brgb}
B_{\rm RG} = B_{\rm MS} \bigg( \frac{r_{\rm MS}}{r_{\rm RG}} \bigg)^2 \, .
\end{equation}
Here, the red giant magnetic field $B_{\rm RG}$ at a mass coordinate $m$ is calculated from the corresponding MS field $B_{\rm MS}$ at the moment when the convective core has its largest extent. The radius $r_{\rm MS}$ is the radial coordinate of the mass shell at this point on the MS, while the radius $r_{\rm RG}$ is the radial coordinate of the mass shell when the star is on the clump. 

The post-MS core magnetic field can also be approximated from the global properties of the star without detailed stellar models, as shown in Appendix \ref{Bcenap}. We find that a reasonable estimate of a red giant's central magnetic field strength, $B_{\rm cen}$, is
\begin{equation}
\label{eqn:Bcen}
B_{\rm cen} \sim L_{\rm MS}^{1/3} M_{\rm c,MS}^{-2/9} \rho_{\rm c,MS}^{-5/18} \rho_{\rm c,RG}^{2/3} \, .
\end{equation}
Here, $L_{\rm MS}$, $M_{\rm c,MS}^{-2/9}$, $\rho_{\rm c,MS}^{-5/18}$, are the MS luminosity, convective core mass, and central density, respectively, and $\rho_{\rm c,RG}$ is the post-MS central density. Figure \ref{fig:DipoleHist} plots $B_{\rm cen}$ in our stellar models. Central magnetic fields of $B_{\rm cen} \sim 10^5 \, {\rm G}$ are expected during the MS, while field strengths of $B_{\rm cen} \sim 10^7 \, {\rm G}$ are expected on the clump if the fields survive to this phase of evolution. In general, field strengths of $B_{\rm RG} > 10^6 \, {\rm G}$ can be expected near the centers of red giant stars.

The spatial structure of a MS dynamo-generated field has important implications for the suppression of oscillation modes during the red giant phase.
%As discussed above, the radial extent of the field will affect the mass range over which the magnetic greenhouse effect is likely to occur.
The angular structure of the field will affect magnetic mode splitting and wave scattering, which could potentially be used to constrain the magnetic field structure within red giants.

While the dynamo is active, we expect the magnetic field to vary on horizontal scales comparable to those of convective eddies, as seen in simulations \citep{Featherstone_2009}. The largest convective eddies at the outer edge of the convective core have length scales of $\sim \! H_P \sim \! r_c$, where $r_c$ is the radius of the convective core. Since these magnetic structures can be long-lived, we expect them to be mostly frozen in to the core when it becomes radiative at the end of the MS, as long as the global field structure remains stable. Therefore, we expect dominant fluctuations in the angular structure of the field to occur on length scales of $\sim \! r_c$, although smaller structures may also exist due to smaller eddies within the convective flow.

To extrapolate these structures into the red giant phase, we assume they remain frozen in place as the core contracts. During this process, the structures shrink by a factor of $r_{\rm RG}/r_{\rm MS}$, where $r_{\rm MS}$ is the radial coordinate of a mass shell on the MS and $r_{\rm RG}$ is its corresponding location in a red giant. Hence, we expect the field to vary on horizontal length scales up to the radial coordinate $r$ in red giant cores. The Ohmic diffusion time across structures of this size is large (see Section~\ref{time}), such that they are not able to be significantly smoothed out within the lifetime of the star.

Horizontal variations break the spherical symmetry of the field and affect its interaction with oscillations. In principle, one could envision a field structure that is tangled only on very small scales $l$, such that its surface resembles the dimpled surface of a cantaloupe and appears nearly spherically symmetric to an incoming wave. For this to happen, the length scale $l$ must be smaller than the {\it radial} wavelength of incoming waves such that $k_r l < 1$. However, we argued above that the field will vary on length scales $r$, and \citet{Fuller_2015} showed that $k_r r \gg 1$ within the cores of red giants for observable oscillation modes. Therefore, realistic field configurations can always break the spherical symmetry of the background and scatter incoming waves into high angular wave numbers $k_\perp$ such that the magnetic greenhouse effect operates as described in \citet{Fuller_2015}.