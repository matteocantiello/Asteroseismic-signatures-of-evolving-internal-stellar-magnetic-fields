\section{Dipole and Quadrupole Mode Visibility}
\subsection{Red Giant Branch}

Fuller et al. (2015) demonstrated that the visibility dipole oscillation modes in RGB stars with strongly magnetized cores is suppressed because wave energy which propagates into the core is lost. Both Fuller et al. (2015) and Stello et al. (2015) found that the amplitudes of depressed dipole oscillation modes are consistent with total wave energy loss in the core, although it remains possible that some small fraction of the wave energy returns to the envelope.

Here, we utilize the same method presented in Fuller et al. (2015) to calculate expected visibilities of both dipole and quadrupole modes in stars with $1.25 \, M_\odot leq M leq 3 \, M_\odot$. The ratio of suppressed mode power to normal mode power is
\begin{equation}
\label{eqn:vsup}
\frac{V_{\rm sup}^2}{V_{\rm norm}^2} = \bigg[ 1 + \Delta \nu \tau T^2 \bigg]^{-1} \, .
\end{equation}
Here, $\Delta \nu$ is the large frequency separation, $\tau$ is the radial mode lifetime, and $T$ is the wave transmission coefficient through the evanescent zone. The value of $T$ can be calculated from
\begin{equation}
\label{eqn:T}
T  = \exp \bigg[ - \int^{r_2}_{r_1} dr \sqrt{ \frac{\big(L_\ell^2 - \omega^2 \big) \big(\omega^2 - N^2 \big)}{v_s^2 \omega^2} \bigg] \, .
\end{equation}