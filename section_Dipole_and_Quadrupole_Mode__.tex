\section{Dipole and Quadrupole Mode Visibility}\label{visibility}


\subsection{Red Giant Branch}
\label{rgb}

%\citet{Fuller_2015} demonstrated that the visibility of dipole oscillation modes in RGB stars with strongly magnetized cores is suppressed because wave energy which propagates into the core is lost. Both \citet{Fuller_2015} and Stello et al. (2015) found that the amplitudes of depressed dipole oscillation modes are consistent with total wave energy loss in the core, although it remains possible that some small fraction of the wave energy returns to the envelope.

We now utilize the same method presented in \citet{Fuller_2015} to calculate expected visibilities of both dipole and quadrupole modes in stars with $1.25 \, M_\odot \leq M \leq 3 \, M_\odot$. The ratio of suppressed mode power to normal mode power is
\begin{equation}
\label{eqn:vsup}
\frac{V_{\rm sup}^2}{V_{\rm norm}^2} = \bigg[ 1 + \Delta \nu \tau T^2 \bigg]^{-1} \, .
\end{equation}
Here, $\Delta \nu$ is the large frequency separation, $\tau$ is the radial mode lifetime, and $T$ is the wave transmission coefficient through the evanescent zone. The value of $T$ can be calculated via
\begin{equation}
\label{eqn:T}
T  = \exp \bigg[ - \int^{r_2}_{r_1} dr \sqrt{ - \frac{ \big( L_\ell^2 - \omega^2 \big) \big(N^2 - \omega^2 \big) }{v_s^2 \omega^2} } \bigg] \, .
\end{equation}
Here, $r_1$ and $r_2$ are the lower and upper boundaries of the evanescent zone, $L_\ell^2 = l(l+1)v_s^2/r^2$ is the Lamb frequency squared, $N$ is the Brunt-Vaisala frequency, $\omega$ is the angular wave frequency, and $v_s$ is the sound speed. We calculate $\Delta \nu$ and the frequency of maximum power $\nu_{\rm max}$ using the scaling relations of \cite{Huber_2011}.

We construct stellar models using MESA (Modules for Experiments in Stellar Evolution) \citep[MESA,release 7456][]{Paxton_2010,Paxton_2013,Paxton_2015} evolving them from the zero age main sequence to the end of core He-burning. The models are non-rotating and adopt the OPAL opacity tables \citep{Iglesias:96} and an initial metallicity of $Z=0.02$ with a mixture taken from \citet{Asplund:2005}.  
Convective regions have been calculated using the mixing-length theory (MLT) with $\alpha_{\rm MLT}=2.0$. The boundaries of convective regions are determined according to the Schwarzschild criterion. Exponentially decaying overshoot at the convective boundaries is included with a mixing parameter $f=0.018$ \citep{2000A&A...360..952H,Paxton_2010}. We include red giant mass-loss using the prescription of \citet{Reimers:1975} with $\eta=0.5$. The inlist used to calculate the models is provided in Appendix~\ref{inlist}.
  
Figure \ref{fig:visibility} shows our predictions for reduced mode power, $(V_{\rm sup}/V_{\rm norm})^2$, as a function of $\nu_{\rm max}$ for stars of various masses as they evolve up the RGB. To first order, the reduced mode power is very similar for stars of different mass. The largest differences occur at low frequencies ($\nu_{\rm max} \lesssim 50 \, \mu {\rm Hz}$), where we predict more massive stars to show lower dipole mode visibilities. However, we caution that clump stars (see Section \ref{clump}) may be difficult to distinguish from RGB stars at these low frequencies.

Figure \ref{fig:visibility} also shows predictions for the reduced power of suppressed quadrupole oscillation modes. Quadrupole modes are expected to exhibit significantly less suppression at all values of $\nu_{\rm max}$. However, for stars low on the RGB ($\nu_{\rm max} \gtrsim 150 \, \mu {\rm Hz}$), we find $(V_{\rm sup}/V_{\rm norm})^2 \lesssim 0.5 $ for quadrupole modes, i.e., substantial mode suppression is expected for quadrupole modes in addition to dipole modes. We predict this quadrupole mode suppression can easily be measured in existing {\it Kepler} data, and can be used to test the magnetic greenhouse hypothesis. 
  
  
  
  
  