\section{Implications for Other Asteroseismic Targets}
\label{pulsators}

\subsection{Magnetic Fields in sdB Stars}

Subdwarf B (sdB) stars are essentially the naked He cores of red clump stars, with masses of $M \simeq 0.47 \, M_\odot$ \citep{fontaine_2012} and thin H envelopes of $M_{\rm H} \sim 10^{-3} \, M_\odot$. They provide an opportunity to constrain much of the physics discussed in this paper, since magnetic fields which would be confined to the cores of clump stars could be visible at the surfaces of sdB stars. If strong fields can be detected in sdB stars, it may indicate that magnetic dipole mode suppression will occur in clump stars. However, \cite{Landstreet_2012} observe no evidence for strong magnetic fields at the surface of any known sdB star, and find that strong fields occur in less than a few percent of sdB stars. 

The lack of strong magnetic fields at the surfaces of sdB stars may have two causes. First, sdB stars evolve primarily from low-mass ($M \lesssim 2.25 \, M_\odot$) stars which have been stripped of their H envelope just prior to the He flash \citep{heber_2009}. Their MS progenitors had convective cores of $M_{\rm core} \lesssim 0.4 \, M_\odot$ (see Figure \ref{fig:DipoleHist}), and therefore any dynamo-generated fields are likely confined to the interiors of sdB stars and are not detectable at their surfaces. Second, sdB stars evolved through a He flash phase, and strong large-scale fields may have been wiped out by convection during that time (see Figure \ref{fig:DipoleHistConv}). Thus, the apparent absence of strong fields at sdB surfaces is not altogether surprising. However, some small fraction of sdBs likely evolve from low-mass magnetic Ap stars, and therefore we may expect to see strong fields (if they are not wiped out by He flashes) at the surfaces of a small percentage of sdB stars. Additional observations, coupled with sdB population synthesis calculations, will be needed to reach a robust conclusion.

Many sdB stars pulsate in p modes (periods of $\sim$minutes) and/or g modes (periods of $\sim$hours). Their pulsations may be used to study magnetic mode alteration in sdBs with strong internal fields. A propagation diagram for an sdB star is shown in Figure \ref{fig:sdBProp}. We find that a magnetic field of $B\sim 10^5 \, {\rm G}$ near the He-H transition (located at $r/R \approx 0.35$ in Figure \ref{fig:sdBProp}) is sufficient to strongly alter a typical sdB g mode with a frequency of $\nu = 10^4 \, \mu {\rm Hz}$. Fields as small as $B\sim 10^3 \, {\rm G}$ near the surface (at $r/R \approx 0.95$) could also create magnetic alteration. Even smaller fields can strongly alter lower frequency modes, although we caution that our conclusions are somewhat sensitive to the mass of the H-envelope and the operation of diffusive/mixing processes. Unlike mixed modes in red giants, g modes in sdB stars are not separated from the surface by a thick evanescent region, and therefore magnetically altered magneto-gravity modes could be detectable at the surface. Therefore, we strongly encourage detailed analyses of the g mode spectra of pulsating sdB stars, as the pulsations may carry information about strong sub-surface fields. 



\subsection{Mode Suppression in Other Types of Pulsators}
  
Since magnetic mode suppression is relatively common in red giant pulsators, it is possible that it operates (but has not been recognized) in other types of pulsators as well. In MS stars, g modes are most vulnerable to magnetic alteration because smaller field strengths are required to suppress lower frequency oscillations (see equation \ref{eqn:Bc}). The magnetic greenhouse effect (as described by \cite{Fuller_2015}) may not operate in the same manner, but strong magnetic fields may still spread the power of oscillation modes into a broad range of spherical harmonics $\ell$ and therefore reduce their visibility. 

It is possible that strong magnetic fields located just outside the core may inhibit the development of large amplitude $\gamma$-Dor pulsations in some stars. In particular, stars passing through the $\gamma$-Dor instability strip at the end of their MS evolution may contain strong magnetic fields that have been deposited in the radiative region around the shrinking convective core. We find that the approximate critical field strength required to inhibit pulsations with a frequency of $\nu = 10 \, \mu {\rm Hz}$ in a $1.6 \, M_\odot$ star passing through the $\gamma$-Dor instability strip is $B_c \approx 10^5 \, {\rm G}$, although the precise value depends somewhat on the value of $N$ (and therefore the mixing processes at work) just outside the core. This field strength is lower than the equipartition fields that could have been deposited during previous MS evolution (we find $B_{\rm eq} \approx 2 \times 10^5 \, {\rm G}$), and it is therefore possible that strong magnetic fields inhibit or alter $\gamma$-Dor pulsation modes in some stars within the instability strip. We also find that more modest fields of $B \sim 10^{3} \, {\rm G}$ are capable of altering g modes near the surface of the star where $\rho$ is much smaller. Therefore, magnetic Ap stars in the $\gamma$-Dor instability strip may exhibit magnetically altered/suppressed g modes. 

Slowly pulsating B (SPB) stars also exhibit g mode pulsations which could be altered by strong internal magnetic fields. We find very similar field characteristics to those in $\gamma$-Dor stars could alter the g modes in a $5 \, M\odot$ SPB model. A field of $\sim 10^5 \, {\rm G}$ just outside the convective core, or a field of $\sim 10^{3} \, {\rm G}$ near the surface would suffice to alter modes of $\nu_{\rm g} = 10 \, \mu {\rm Hz}$. SPB stars with strong internal and/or surface fields thus present another opportunity to examine g mode interactions with magnetic fields.


  
  
  
  
  
  
  
  
  
  
  
  
  
  
  