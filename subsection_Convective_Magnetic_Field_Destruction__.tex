\subsection{Convective Magnetic Field Destruction}

The predictions outlined above assumed that magnetic fields generated by cored dynamos during the main sequence are preserved through red giant evolution. While this is likely a good assumption for magnetic fields in radiative regions (see Section \ref{time}), it may not hold in regions of the star that become convective during post-MS evolution. In fact, convection may destroy dynamo-generated fields during later phases of evolution, especially if the newly formed convective regions are slowly rotating (as measured in many red giant cores, \cite{Mosser_2012}), such that a large scale dynamo does not operate.

We emphasize that it is not clear whether convection will destroy pre-existing stable field configurations. If the convective energy density, $\epsilon_{\rm con}$, is larger than the magnetic energy density, $\epsilon_{\rm mag}$, the convective motions may not be constrained by the magnetic fields. One may therefore expect the convective motions to scramble the pre-existing field into an unstable configuration. 

Need a sentence or two here about convective destruction of fields. 

As shown in Figure \ref{fig:DipoleHistConv}, convective regions occupy large fractions of red giant stars. During the first dredge-up, the convective envelope extends down to mass coordinates of $ \sim 0.25 \! \, M_\odot$ in low-mass stars. It is therefore possible that any fields at mass coordinates above this region are destroyed by the convective envelope.

Of particular importance are the convective regions that develop during helium flashes in stars of $M \lesssim 2.25 \, M_\odot$. Figure \label{fig:DipoleHistConv} shows that these flashes induce convection in all mass coordinates below $ \sim \! 0.4 \, M_\odot$ in low-mass stars. We find that stars with $M \lesssim 2.1 \, M_\odot$ have evolved such that {\it all} mass coordinates  of the radiative core on the clump were convective at some point of prior red giant evolution. 

This detail allows us to make a second prediction about the occurrence of magnetic mode suppression in clump stars. If convection in red giants destroys previously existing fields, we expect dipole mode suppression to only occur in relatively massive $M \gtrsim 2.1 \, M_\odot$ clump stars. Only these massive stars contain radiative regions that were convective on the MS (and may contain strong dynamo-generated fields) but which were not convective on the RGB. 

The observation (or lack thereof) of dipole mode suppression in clump stars will therefore provide great understanding of magnetic field evolution in stellar interiors. If mode suppression is common in stars with masses as small as $\sim \! 1.5 \, M_\odot$, this would indicate that magnetic fields are robust, and are able to survive through post-MS convective phases. If mode suppression only occurs in stars with $M \gtrsim 2.1 \, M_\odot$, this would indicate that post-MS convection generally destroys pre-existing fields. 

We caution that these results can be somewhat influenced by the size of the convective core, which may not be accurately calculated by stellar evolution codes such as MESA, most likely because of mixing induced by convective overshoot. Indeed, asteroseismic studies of clump stars (\cite{montalban_2013,stello_2013,mosser_2014,bossini_2015,constantino_2015}) and sub-dwarf B stars \citep{vangrootel_2010a,vangrootel_2010b,charpinet_2011,Schindler_2015} indicate that the convective core is significantly larger than it is in stellar evolution codes, even using optimistic overshooting prescriptions. Main sequence convective cores may also be somewhat larger and show evidence for enhanced mixing \citep{moravveji_2015}.  Our calculations of mode visibilities and field strengths $B_c$ are not strongly affected, but inferences based on the extent of the convective core become less certain.
%In particular, the large convective core masses inferred for clump/sdB stars may contribute to lower dipole mode suppression rates in these stars if 
    
    