

\subsection{Magnetic Fields in White Dwarfs}

Many white dwarfs (WDs) exhibit strong ($B \gtrsim 3 \, {\rm MG}$) surface magnetic fields. The exact fraction of WDs which have strong fields is debated but is of oder 10\% \citep{Hollands_2015}. Moreover, there is a dearth of magnetic WDs with field strengths below $10 \, {\rm MG}$, and the magnetic WDs are systematically more massive \citep{Ferrario_2015B}.

Our work suggests that some of the strong fields observed at WD surfaces could be the remnants of MS core-dynamo-generated fields. Equipartition fields of $\sim \! 2 \times 10^5 \, {\rm G}$ generated by a core dynamo would evolve into fields of $\sim 2 \times 10^7 \, {\rm G}$ if their flux is conserved from the MS to the WD phase. Thus, the field strengths inferred for core-dynamo-generated fields are squarely within the observed distribution of WD surface fields \citep{Ferrario_2015A,Ferrario_2015B}.

Figure \ref{fig:DipoleHist} shows that the core convective region is restricted to mass coordinates below $\approx 0.5 \, M_\odot$ for stars of $M \lesssim 2.5 M_\odot$. Using the MS-WD mass relation of \cite{Renedo_2010}, these low-mass stars account for the majority of WDs, whose mass distribution peaks near $0.6 \, M_\odot$ \citep{Rebassa-Mansergas_2015}. Since the ohmic diffusion time is very long in WDs \citep{Ferrario_2015B}, we expect that core-dynamo-generated fields are unlikely to be visible at the surface of most WDs. 

However, in stars of $M \gtrsim 3 \, M_\odot$, the convective core extends to mass coordinates larger than $0.6 \, M_\odot$.  In these stars, the entire mass of the WD descendant lies within the mass coordinate occupied by the MS convective core. Therefore, strongly magnetized regions may extend all the way to the WD surface where they can be observed. Stars over $M \gtrsim 3 \, M_\odot$ produce WDs of $M_{\rm WD} \gtrsim 0.7 \, M_\odot$, similar to the typical masses of magnetic WDs \citep{Ferrario_2015B}. Thus, some magnetic WDs may be magnetized due to MS convective core dynamos, which could partially explain why magnetic WDs are more massive on average. However, it remains possible that magnetic WDs are the descendants of magnetic Ap stars, or that they are formed through WD mergers. 

The interesting corollary to this discussion is that strong magnetic fields may exist within the interiors of most WDs even though the fields are not visible at the surface. Indeed, based on the MS-WD mass relation, most WDs originate from progenitors of $M \gtrsim 1.5 \, M_\odot$ in which core dynamos operate and produce strong magnetic fields in over 50\% of RGB stellar cores (Stello et al. 2015). We therefore speculate that many (perhaps the majority of) WDs could contain strong ($B \gtrsim 10^6 \, {\rm G}$) magnetic fields which are confined within the stellar interior and not detectable at the surface. These magnetic fields may have very important implications for WD evolution, and for the outcome of WD mergers. 


\subsubsection{Helium White Dwarfs}

Helium white dwarfs (He WDs) typically have masses in the range $0.15 \, M_\odot \lesssim M \lesssim 0.4 \, M_\odot$ and are formed when a companion star strips the hydrogen envelope of the He WD progenitor as it ascends the RGB (***REF?***). He WDs are essentially the naked cores of the RGB stars analyzed in \cite{Fuller_2015} and Stello et al. 2015. Unless internal magnetic fields are somehow destroyed by envelope mass loss, we expect some He WDs to exhibit surface fields of $B \gtrsim 10^5 \, {\rm G}$. As far as we are aware, strong magnetic fields have not yet been observed at the surfaces of any He WDs, even though they may be detectable.

Predicting the fraction of He WDs that will exhibit strong surface fields is not straightforward, as it depends both on the progenitor mass and the He WD mass. For instance, the $1.75 \, M_\odot$ model shown in Figure \ref{fig:DipoleHist} has an MS convective core that extends to a mass coordinate of $\approx 0.24 \, M_\odot$. Therefore, its He WD descendant may only exhibit strong surface fields if its mass is $M_{\rm WD} \lesssim 0.24 \, M_\odot$, otherwise the fields may remain buried. We encourage searches for magnetic fields in He WDs, as their detection would allow further characterization of the strong fields inferred to exist within red giant cores.



\subsection{Magnetic Fields in Neutron Stars}

Since the observations of Stello et al. 2015 show that core-dynamo-generated magnetic fields frequently survive well into RGB evolution in low-mass stars, it is possible that these fields are also long-lived in massive stars that spawn neutron stars upon their death. We find typical equipartition field strengths of $B \sim 10^6 \, {\rm G}$ in the MS convective core of $M \sim 12 \, M_\odot$ neutron star progenitors. Flux conservation of the field within the inner $1.4 \, M_\odot$ (which has a radius of $\sim \! 0.5 \, R_\odot$ on the MS) to a neutron star radius of $12 \, {\rm km}$ would lead to neutron star surface field strengths of $\sim \! 10^{15} \, {\rm G}$, i.e., magnetar field strengths.  

The magnetar birth rate is highly uncertain, with plausible estimates in the range of (\cite{keane_1998,mereghetti_2015}) $5-50\%$ of the neutron star birth rate. It is therefore possible that stable magnetic fields in the cores massive stars are just as common as in low-mass stars, leading to magnetar birth rates of order $50 \%$ the neutron star birth rate.  
%One possibility is that the core-dynamo-generated fields are less likely to settle into a stable configuration in massive stars compared to their lower mass counterparts.
If magnetar birth rates turn out to be smaller, it may indicate that post-MS core convective phases (He, C, Ne, O, or Si burning) destroy MS core-dynamo-generated fields and prevent magnetar formation, as we have hypothesized to happen in low-mass stars. 
%This is certainly possible, especially if angular momentum transport is efficient in massive stars (as it appears to be in low-mass stars) such that their cores are slowly rotating in post-MS phases of evolution and are not able to sustain a rotation-driven dynamo.
An absence of dipole mode suppression in low-mass clump stars would support this hypothesis. It also remains possible that most magnetars are the descendants of magnetic OB stars, or that their fields are generated during a proto-neutron star dynamo \cite{1992ApJ...392L...9D}.



\subsection{Magnetic Fields in sdB Stars}

Subdwarf B (sdB) stars are essentially the naked He cores of red clump stars, with masses of $M \simeq 0.47 \, M_\odot$ \citep{fontaine_2012} and thin H envelopes of $M_{\rm H} \sim 10^{-3} \, M_\odot$. They provide an opportunity to constrain much of the physics discussed in this paper, since magnetic fields which would be confined to the cores of clump stars could be visible at the surfaces of sdB stars. If strong fields can be detected in sdB stars, it may indicate that magnetic dipole mode suppression will occur in clump stars. However, \cite{Landstreet_2012} observe no evidence for strong magnetic fields at the surface of any known sdB star, and find that strong fields occur in less than a few percent of sdB stars. 

The lack of strong magnetic fields at the surfaces of sdB stars may have two causes. First, sdB stars evolve primarily from low-mass stars ($M \lesssim 2.25 \, M_\odot$) which have been stripped of their H envelope just prior to the He flash \citep{heber_2009}. Their main sequence progenitors had convective cores of $M_{\rm core} \lesssim 0.4 \, M_\odot$ (see Figure \ref{fig:DipoleHist}), and therefore any dynamo-generated fields are likely confined to the interiors of sdB stars and are not detectable at their surfaces. Second, sdB stars evolved through a He flash phase, and strong large-scale fields may have been wiped out by convection during that time (see Figure \ref{fig:DipoleHistConv}). Thus, the apparent absence of strong fields at sdB surfaces is not altogether surprising. However, some small fraction of sdBs likely evolve from low-mass magnetic Ap stars, and therefore we may expect to see strong fields (if they are not wiped out by He flashes) at the surfaces of a small percentage of sdB stars. Additional observations, coupled with sdB population synthesis calculations, will be needed to reach a robust conclusion.

Many sdB stars pulsate in p modes (periods of $\sim$minutes) and/or g modes (periods of $\sim$hours). Their pulsations may be used to study magnetic mode alteration in sdBs with strong internal fields. A propagation diagram for an sdB star is shown in Figure \ref{fig:sdBProp}. We find that a magnetic field of $B\sim 10^5 \, {\rm G}$ near the He-H transition (located at $r/R \approx 0.35$ in Figure \ref{fig:sdBProp}) is sufficient to strongly alter a typical sdB g mode with a frequency of $\nu = 10^4 \, \mu {\rm Hz}$. Fields as small as $B\sim 10^3 \, {\rm G}$ near the surface (at $r/R \approx 0.95$) could also create magnetic alteration. Even smaller fields can strongly alter lower frequency modes, although we caution that our conclusions are somewhat sensitive to the mass of the H-envelope and the operation of diffusive/mixing processes. Unlike mixed modes in red giants, g modes in sdB stars are not separated from the surface by a thick evanescent region, and therefore magnetically altered magneto-gravity modes could be detectable at the surface. Therefore, we strongly encourage detailed analyses of the g mode spectra of pulsating sdB stars, as the pulsations may carry information about strong sub-surface fields. 



\subsection{Mode Suppression in Other Types of Pulsators}
  
Since magnetic mode suppression is relatively common in red giant pulsators, it is possible that it operates (but has not been recognized) in other types of pulsators as well. In MS stars, g modes are most vulnerable to magnetic alteration because smaller field strengths are required to suppress lower frequency oscillations (see equation \ref{eqn:Bc}). The magnetic greenhouse effect (as described by \cite{Fuller_2015}) may not operate in the same manner, but strong magnetic fields may still spread the power of oscillation modes into a broad range of spherical harmonics $\ell$ and therefore reduce their visibility. 

It is possible that strong magnetic fields located just outside the core may inhibit the development of large amplitude $\gamma$-Dor pulsations in some stars. In particular, stars passing through the $\gamma$-Dor instability strip at the end of their MS evolution may contain strong magnetic fields that have been deposited in the radiative region around the shrinking convective core. We find that the approximate critical field strength required to inhibit pulsations with a frequency of $f = 10 \, \mu {\rm Hz}$ in a $1.6 \, M_\odot$ star passing through the $\gamma$-Dor instability strip is $B_c \approx 10^5 \, {\rm G}$, although the precise value depends somewhat on the value of $N$ (and therefore the mixing processes at work) just outside the core. This field strength is lower than the equipartition fields that could have been deposited during previous MS evolution (we find $B_{\rm eq} \approx 2 \times 10^5 \, {\rm G}$), and it is therefore possible that strong magnetic fields inhibit or alter $\gamma$-Dor pulsation modes in some stars within the instability strip. We also find that more modest fields of $B \sim 10^{3} \, {\rm G}$ are capable of altering g modes near the surface of the star where $\rho$ is much smaller. Therefore, magnetic Ap stars in the $\gamma$-Dor instability strip may exhibit magnetically altered/suppressed g modes. 

Slowly pulsating B (SPB) stars also exhibit g mode pulsations which could be altered by strong internal magnetic fields. We find very similar field characteristics to those in $\gamma$-Dor stars could alter the g modes in a $5 \, M\odot$ SPB model. A field of $\sim 10^5 \, {\rm G}$ just outside the convective core, or a field of $\sim 10^{3} \, {\rm G}$ near the surface would suffice to alter modes of $\nu_{\rm g} = 10 \, \mu {\rm Hz}$. SPB stars with strong internal and/or surface fields thus present another opportunity to examine g mode interactions with magnetic fields.


  
  
  
  
  
  
  
  
  
  
  
  
  
  
  
  
  
  
  
  
  
  
  
  
  
  