%\section{Discussion}
%\begin{itemize}
%\item Rotational suppression
%\item J-transport
%\item SdB Stars
%\item Role of magnetic fields in shaping Planetary Nebulae
%\item Magnetized remnants
%\item OB stars (Add refs to FORS, MIMEs...)
%\item Type Ia SN (all WD have buried internal fields?)
%\item Fossil Fields
%\end{itemize}
\section{Implications for RGB Stars}
\subsection{Incidence Rate of Strong Magnetic Fields in Red Giant Stars}
\cite{Stello_2016} determined the incidence of strong internal magnetic fields as function of stellar mass in red giants. Stars that did not have convective cores during the MS ($M<1.1\mso$) do not show suppressed dipole modes, while red giants with $1.1 \, M_\odot \lesssim M \lesssim 1.5 \, M_\odot$ exhibit an increasing incidence in mode suppression rate with mass. In Section \ref{rgb} we discussed how the extension of the MS convective core could help create this trend. However, this does not explain why the incidence rate of depressed dipole modes appears to saturate at $\approx 50\%$ in red giants with $M \gtrsim 1.5 \mso$.

%One possibility is that there is a range of efficiencies in converting kinetic energy density into magnetic energy density in the convective cores of intermediate mass stars. This could be due, for example, to a distribution of core rotation rates during the MS. 
One important consideration is the increasing efficiency of dynamo action with rotation rate. Stars with relatively rapidly rotating convective cores (small Rossby numbers, see Sec.~\ref{msdynamo}) would generate strong core fields, while stars with slowly rotating cores might not produce strong and/or stable internal magnetic fields. Indeed, several recent studies of dynamo action in young solar type stars \citep{See_2015,Vidotto_2014,Folsom_2016} have shown that their dynamos produce fields that saturate for $Ro \lesssim 0.1$, and which scale as $B \propto Ro^{-1}-Ro^{-1.5}$ for $Ro \gtrsim 0.1$. Since the core convective turnover time scales for stars with $1.2 \, M_\odot \lesssim M \lesssim 2 \, M_\odot$ are roughly one month, we expect substantially reduced field strengths in the cores of stars which are spun down to rotation periods of $P \gtrsim 3 \, {\rm d}$ by the end of the MS. Indeed, stars of $M\lesssim 1.3 \, M_\odot$ lie below the Kraft break and can likely be spun down to such rates by the end of the MS. \citep[see][]{VanSaders_2013}. Their dynamos may then shut down, and this may help explain the increasing prevalence of strong magnetic fields with stellar mass in the $1.1 \, M_\odot$-$1.5 \, M_\odot$ mass range.

%For this to work, there should be a large population of MS A-type stars rotating with periods similar to or larger than the convective turnover timescale in the core. The latter is on the order of a month, whereas the observed typical surface rotation periods of A stars is a few days \citep{Zorec_2012}.
%with  asteroseismology indicating that these stars are likely to rotate almost rigidly during the MS \citep[e.g.][]{Benomar_2015}.
%This scenario seems unlikely, unless stars in the $1.1$-$1.4 \, M_\odot$ range can be magnetically braked to rotation periods of weeks or longer before the end of the MS \citep[see][]{VanSaders_2013}. Their dynamos may then shut down, and this may help explain the increasing prevalence of strong magnetic fields in this mass range.

Another possibility is that, at the end of the MS and its associated dynamo, the magnetic field is able to evolve into a stable configuration only in $\approx 50\%$ of the cases. The evolution of magnetic fields into stable magnetic configurations has been studied \citep{Braithwaite_2006}, but it is difficult to make detailed predictions as the outcome depends on the complex initial conditions of the magnetic configuration left by turbulent convection, in particular magnetic energy and magnetic helicity \citep{Braithwaite_2008}. Moreover, these theoretical calculations do not include a compositional stratification, which instead might play an important role in confining the magnetic field after dynamo action ceases in the convective cores of stars. 