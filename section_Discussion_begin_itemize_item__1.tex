\section{Discussion}
\begin{itemize}
%\item Rotational suppression
%\item J-transport
%\item SdB Stars
\item Role of magnetic fields in shaping Planetary Nebulae
%\item Magnetized remnants
\item OB stars (Add refs to FORS, MIMEs...)
%\item Type Ia SN (all WD have buried internal fields?)
%\item Fossil Fields
\end{itemize}

\subsection{Incidence Rate of Strong Magnetic Fields in Red Giant Stars}
Stello et al. 2015 determined the incidence of strong internal magnetic fields as function of stellar mass in red giants. Stars that did not have convective cores during the main sequence ($M<1.1\mso$) do not show depressed mixed modes, while red giants with mass above this limit indicates a gradually increasing incidence rate of strong internal magnetism.    In Sec.~\ref{rgb} we have discussed the extension of the hydrogen convective core as a possible reason for this slow ramp up. However this does not explain why the incidence rate of depressed mixed modes only reaches 50-60\% in red giants with mass around $2\mso$.
A possibility is that there is a range of efficiencies in converting kinetic energy density into magnetic energy density in the convective cores of intermediate mass stars. This could be due, for example, to the distribution of rotation rates during the main sequence. Stars with relatively rapidly rotating convective cores (small Rossby numbers) would result in efficient dynamo action, while stars with slow-rotating cores might not produce strong internal magnetic fields. For this to work there should be a large population of main sequence intermediate stars with cores rotating with periods similar or larger than the convective turnover timescale. The latter is on the order of months. The  observed typical surface rotation periods of A stars is a few days \citep{Zorec_2012}, with  asteroseismology indicating that these stars are likely to rotate almost rigidly during the main sequence \citep[e.g.][]{Benomar_2015}. Therefore this scenario is probably unlikely.

Another option is that, at the end of the main sequence and its associated contemporary dynamo, the magnetic field is able to evolve into a stable configuration only in 50-60\% of the cases. The evolution of magnetic fields into stable magnetic configurations has been studied \citep{Braithwaite_2006}, but it is difficult to make detailed predictions as the outcome depends on the complex initial conditions of the magnetic configuration left by turbulent convection, in particular magnetic energy and magnetic helicity \citep{Braithwaite_2008}. Moreover, all theoretical calculations have been performed in the absence of a compositional stratification, which instead might play an important role in confining the magnetic field evolution once dynamo action ceases in the convective cores of stars. 