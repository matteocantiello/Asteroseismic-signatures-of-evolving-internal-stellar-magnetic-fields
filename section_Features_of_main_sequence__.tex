\section{Features of main sequence dynamo-generated fields}

\subsection{Operation of Main-Sequence Dynamo}

The main sequence of stars more massive than about 1.1$\mso$ is characterized by the presence of a convective core,
%This is due to central temperatures high enough to activate the CNO cycle, characterized by a set of reaction rates depending more steeply on temperature than the p-p chain ($\epsilon_{\rm pp}\propto \rho T^{4}$, while $\epsilon_{\rm CNO}\propto \rho T^{18}$ ).
%Due to the high conductivity of stellar plasmas, 
%in the presence of rotation
which is expected to host a magnetic dynamo. Dynamo action converts a fraction of the kinetic energy 
of the convective motions into magnetic energy, with magnetic fields sustained against dissipation  \citep[see e.g.,][]{Brandenburg_2005}. The amplitude and scale of the generated magnetic field depends on the relative importance of rotation, which is usually quantified by the Rossby number $Ro$. The Rossby number is the ratio between inertial and Coriolis forces, which can be quantified by the ratio between the local rotation period and turnover timescale, $Ro = P_{\rm rot}/(2 t_{\rm con})$. 
%For fast rotation (Ro<1) both dynamo theories and observations expect efficient dynamo resulting in large scale magnetic fields with %amplitude close to equipartition.
Efficient dynamo action is expected for $Ro \lesssim 1$. In this case, the field is expected to be large scale and with an amplitude corresponding to equipartition between the magnetic energy density ($B^2/8\pi$) and the kinetic energy density in the flow ($\rho v^2_{\rm con}/2$), where $v_{\rm con}$ is an RMS convective velocity.

Magneto-hydrodynamics simulations of the central regions of a $2\mso$ A-type star rotating between
1 and 4 times the solar mean angular velocity (rotation periods of 28 and 7 days) show dynamo action 
with magnetic fields reaching a considerable fraction of equipartition ($B \approx 10^4-10^5$ G) \citep{Brun_2005}. Typical values for the surface rotation periods of A stars are shorter \citep[about 1 day, see e.g.]{Zorec_2012}. Convective turnover timescales within the core are generally larger, with $t_{\rm con} = 2 \alpha H_P/v_{\rm con} \sim 1 \, {\rm month}$ for a $1.5 \, M_\odot$ model. Here, $\alpha H_P$ is the mixing length and $H_P$ is a pressure scale height, and we evaluate $v_{\rm con}$ from mixing length theory (MLT) with $\alpha = 2$. 

Moreover, asteroseismic observations of slowly rotating A stars suggest these stars are nearly rigidly rotating \cite{Kurtz_2014}, and very little angular momentum is lost for stars above $1.3\mso$ \citep[Kraft break, see e.g.]{1967ApJ...150..551K,2013ApJ...776...67V}. We therefore expect $Ro \ll 1$ in the cores of stars above $1.1-1.3\mso$. Note however that smaller scale, smaller amplitude magnetic fields can be still generated in the absence of rapid rotation, so sizable magnetic fluxes might well be ubiquitous in stellar convective cores. 
%This is usually referred as ``equipartition'' field.


\subsection{Timescales}
\label{time}

Here we discuss the relevant timescales affecting the evolution of internal magnetic fields and their ability to 
suppress oscillation modes through a magnetic greenhouse effect \citep{Fuller_2015}.
We assume that at the end of the MS a magnetic field is present below the maximum Lagrangian mass occupied by the convective core during core H-burning.

The first important timescale is the Ohmic timescale $t_{\rm Ohm}= H_{\rm P}^2/\eta$, the time it takes a stable magnetic field in a radiative region to diffuse across a pressure scale height $H_{\rm P}$. This timescale is usually quite long, due to the small values of the magnetic diffusivity $\eta$ in stellar plasmas. Figure~\ref{fig:timescales} shows that in the core of a $1.5\mso$ evolving from the MS to the RGB, $t_{\rm Ohm}$ varies between $10^8-10^{12}$ years. Therefore we can safely assume that magnetic fields present in the stellar core at the end of the main sequence are frozen in their Lagrangian mass coordinate.
Note that  $t_{\rm Ohm}$ does not depend on the amplitude or geometry of the magnetic field, but only on the local value of the magnetic diffusivity. The magnetic diffusivity is the inverse of the electrical conductivity,  which in RGB stars has to be calculated carefully as certain regions are partially/fully degenerate. Moving from non-degenerate, to partially and fully degenerate regions, we calculate the magnetic diffusivity according to \cite{1968dms..book.....S}, \cite{1987ApJ...313..284W} and \cite{1984MNRAS.209..511N} respectively, applying a smooth interpolation in the transition regions.

The other important timescale is the H-shell burning timescale. As a star moves from the end of its main sequence to the RGB phase, the ashes of H-shell burning increase the size of its He core. We can write the timescale of this process as $t_{\rm Acc} = H_{\rm P} 4\pi r^2 \rho / \dot{m}$, where $r$ is the local radial coordinate and $\dot{m}$ is the He accretion rate. If $t_{\rm Acc} < t_{\rm Ohm}$, then the magnetic field can be buried below the He raining from the H-shell. Figure~\ref{fig:timescales} shows that in a $1.5\mso$ star this is always the case. As a consequence, if the He core grows substantially during the sub-giant/early RGB phases, magnetic fields can be efficiently buried below the H-shell, the location where the waves are most sensitive to the magnetic greenhouse effect (see e.g. the $1.25\mso$ model in Fig.~\ref{fig:DipoleHist}).
%Therefore we expect a decrease of magnetic suppression effects for decreasing initial stellar mass in red giants in the mass range $1.1-1.75\mso$.


Therefore during the main sequence large scale magnetic fields in stellar cores are not expected to diffuse outside and reach the surface, due to the fact that for these stars the Ohmic diffusion timescale 
%($\tau_{\rm Ohm}\sim \rso^2/\eta$, with $\eta$ the magnetic diffusivity)
is longer than the stellar lifetime. \citet{MacGregor_2003} discuss the possibility that magnetic buoyancy instabilities can bring small, magnetized fibrils at the stellar surface. However the inclusion of realistic compositional gradients seems to disfavor this scenario, increasing considerably the timescales of magnetic buoyancy \citep{MacDonald_2004}.

%\subsection{Magnetic buoyancy}


%- Dynamo Action

%- Simulations (Brun)

%- Role of Rossby Numbers

%- Kraft Break
  
%- Typical Ro and Beq   
 
  
  
  
  
  
  