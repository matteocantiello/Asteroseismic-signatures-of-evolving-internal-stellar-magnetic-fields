\section{Features of main sequence dynamo-generated fields}

\subsection{Operation of Main-Sequence Dynamo}

The main sequence of stars more massive than about 1.1$\mso$ is characterized by the presence of a convective core.
This is due to central temperatures high enough to activate the CNO cycle, characterized by a set of reaction rates depending more steeply on temperature than the p-p chain ($\epsilon_{\rm pp}\propto \rho T^{4}$, while $\epsilon_{\rm CNO}\propto \rho T^{18}$ ).

Due to the highly conductive nature of stellar plasmas, 
%in the presence of rotation
these cores are expected to host a contemporary magnetic dynamo. Dynamo action converts a fraction of the kinetic energy 
of the convective motions into magnetic energy, with magnetic fields sustained against dissipation  \citep[see e.g.,][]{Brandenburg_2005}. The amplitude and scale of the generated magnetic field depends on the relative importance of rotation, which is usually quantified by the Rossby number $Ro$. The Rossby number is the ratio between inertia and Coriolis forces, which can be quantified by the ratio between the local rotation period and turnover timescale. 
%For fast rotation (Ro<1) both dynamo theories and observations expect efficient dynamo resulting in large scale magnetic fields with %amplitude close to equipartition.
Efficient dynamo action is expected for $Ro<1$. In this case the field is expected to be large scale and with an amplitude corresponding to equipartition between the magnetic energy ($B^2/8\pi$) and the kinetic energy in the flow ($\rho v^2_{\rm conv}/2$). 


Magneto-hydrodynamics calculations of the central regions of a $2\mso$ A-type star rotating with 
1 and 4 times the solar mean angular velocity (rotation periods of 28 and 7 days) show dynamo action 
with magnetic fields reaching a considerable fraction of equipartition \citep{Brun_2005}. Typical values for the surface  rotation periods of A stars are shorter \citep[about 1 day, see e.g.]{Zorec_2012}. Moreover asteroseismic observations of slowly rotating A stars suggest these stars might be rigidly rotating \cite{Kurtz_2014}. Finally very little angular momentum is lost for stars above $1.3\mso$ \citep[Kraft break, see e.g.]{1967ApJ...150..551K,2013ApJ...776...67V}. Therefore we expect Rossby numbers < 1 in the cores of stars above 1.1-1.3 $\mso$.
Note however that smaller scale, smaller amplitude magnetic fields can be generated also in the absence of rapid rotation, so sizable magnetic fluxes might well be ubiquitous in all stellar convective cores. 
%This is usually referred as ``equipartition'' field.

Large scale magnetic fields in stellar cores are not expected to diffuse outside and reach the surface, due to the fact that for these stars the Ohmic diffusion timescale ($\tau_{\rm Ohm}\sim \rso^2/\eta$, with $\eta$ the magnetic diffusivity) is longer than the stellar lifetime. \citet{MacGregor_2003} discuss the possibility that magnetic buoyancy instabilities can bring small, magnetized fibrils at the stellar surface. However the inclusion of realistic compositional gradients seems to disfavor this scenario, increasing considerably the timescales of magnetic buoyancy \citep{MacDonald_2004}.  




- Dynamo Action

- Simulations (Brun)

- Role of Rossby Numbers

- Kraft Break
  
- Typical Ro and Beq   
 
  
  
  
  
  
  