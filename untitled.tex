\subsection{Extent of Magnetic Fields and Magnetic Mode Suppression}
\label{rgb}

The strong magnetic fields created by a core-dynamo will be mostly confined to the convective core. Magnetohydrodynamical simulations show that the field strength drops off rapidly in overlying regions (\citealt{Featherstone_2009}, Augustson et al. 2015). Moreover, we show in Section \ref{time} that the field will not be able to substantially diffuse outward during the lifetime of the star. 

The size of the convective core changes significantly during main-sequence evolution, as shown in Figure \ref{fig:DipoleHist}. Here, we have used the Schwarzschild criterion to determine the extent of the convective region, as appropriate for stars in this mass range (Moore et al. 2015). Generally, the mass contained within the convective core decreases as stars approach core hydrogen depletion. However, this does not entail that the extent of strongly magnetized regions decreases, because any region which is convective at some point during the main sequence may contain strong fields previously deposited by the dynamo action. This assertion is consistent with simulations \citep{Featherstone_2009} that indicate that convective core dynamos do not destroy overlying magnetic fields, and with the detection of magnetic fields long after the termination of main-sequence dynamos (Stello et al. 2015).

During post-main-sequence evolution, we therefore expect that strong fields will exist only within regions that were convective at some point on the main-sequence, indicated by the pink shaded region in Figure \ref{fig:DipoleHist}. These fields are most likely to lead to magnetic suppression on the RGB if they exist at the mass coordinate of the hydrogen burning shell. Moreover, the suppression will only be evident for stars in the sub-giant/lower RGB phase of evolution, approximately in the range $50 \, \mu {\rm Hz} \lesssim \nu_{\rm max} \lesssim 500 \, \mu {\rm Hz}$.

Figure \ref{fig:DipoleHist} can now be used to understand the occurrence of magnetic mode suppression. Suppression is most likely to be observed when the red line lies within a pink shaded region and between the vertical dashed lines. Magnetic suppression is unlikely to be observed in stars with $M \lesssim 1.5 \, M_\odot$ because the H-burning shell lies above the magnetized (pink) regions during the RGB. Magnetic suppression is much more common in stars with $M \gtrsim 1.5 \, M_\odot$, for which the H-burning shell lies within magnetized regions on the lower RGB. We propose that this feature of stellar evolution accounts for the sharp rise in magnetic suppression for masses $M \gtrsim 1.5 \, M_\odot$ found by Stello et al. 2015.

    
    
  
  