




\section{Magnetic Mode Splitting}
\label{magmode}

Here we provide an approximate formula to estimate the magnitude of magnetic splitting from equation \ref{eqn:dommag}. To simplify the calculation, we adopt an unphysical scenario of a magnetic field purely in the radial direction. In this scenario the splitting is independent of $m$, the more realistic case of a dipolar field is treated in \cite{Unno_1989} (but beware of typos in equation 19.65), and can produce different splittings for different $m$ modes depending on the angle between the magnetic and rotation axes.

For a radially symmetric field, equation \ref{eqn:dB} reduces in the WKB limit to 
\begin{equation}
\delta {\boldsymbol B} \simeq i B_r k_r \boldsymbol{\xi}_\perp \, ,
\end{equation}
where $\boldsymbol{\xi}_\perp = \xi_\perp r \boldsymbol{\nabla} Y_{lm}$ is the horizontal wave displacement. Equation \ref{eqn:dommag} then yields
\begin{equation}
\label{eqn:dommag2}
\frac{\delta \omega_{\rm M}}{\omega} \simeq \frac{\ell(\ell+1)}{8 \pi \omega^2 } \frac{\int r^2 k_r^2 \xi_\perp^2 B_r^2 d r }{\int  \rho r^2 \Big[ \xi_r^2 + \ell(\ell+1) \xi_\perp^2 \Big] d r }\, .
\end{equation}
For a mode with most of its inertia in the g mode cavity, this can be written 
\begin{equation}
\label{eqn:dommag3}
\frac{\delta \omega_{\rm M}}{\omega} \sim \frac{1}{8} \frac{\int r^2 \xi_\perp^2 \bigg(\frac{B_r}{B_c}\bigg)^2 d r }{\int \rho r^2 \xi_\perp^2 d r }\, ,
\end{equation}
where we have used the definition of $B_c$ from equation \ref{eqn:Bc}. Pressure dominated modes will exhibit slightly smaller magnetic splitting due to their inertia in the envelope.

Next, we realize that in the WKB limit, the envelope of the quantity $\rho r^2 \omega^2 v_{{\rm g},r} \xi_\perp^2$ (which represents an energy flux) is constant. Here $v_{{\rm g},r}$ is the gravity wave group velocity in the radial direction, $v_{{\rm g},r} = \omega^2 r/\sqrt{\ell(\ell+1)N^2}$. Then we have
\begin{equation}
\label{eqn:dommag4}
\frac{\delta \omega_{\rm M}}{\omega} \sim \frac{1}{8} \frac{\int v_{{\rm g},r}^{-1} \bigg(\frac{B_r}{B_c}\bigg)^2 d r }{\int v_{{\rm g},r}^{-1} d r }\, .
\end{equation}
The denominator is directly related to the g mode period spacing $\Delta P_{\rm g}$ (see equation 12 of \cite{Chaplin_2013}),
\begin{equation}
\label{eqn:dpg}
\int v_{{\rm g},r}^{-1} dr = \frac{2 \pi^2}{\omega^2 \Delta P_{\rm g}} = \frac{1}{2 \Delta \nu_{\rm g}} \, ,
\end{equation}
and $\Delta \nu _{\rm g}$ is the associated frequency splitting. After inserting this expression into equation \ref{eqn:dommag4}, a little rearranging yields equation \ref{eqn:dfmag}. We emphasize that for a realistic field configuration, the frequency perturbation will depend on both $\ell$ and $m$ (allowing modes to be magnetically split), and may be different by a factor of a few.

