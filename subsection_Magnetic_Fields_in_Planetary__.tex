\subsection{Magnetic Fields in Planetary Nebulae}
A large fraction of planetary nebulae is characterized by bipolar or other non-spherically symmetric shapes. 
One of the possible mechanisms for this lack of symmetry is the presence of a magnetic field in the central star \citep[See e.g.][]{Garc_a_Segura_1997,Chevalier_1994}. The magnetic fields generated by dynamos in the convective cores of main sequence stars above $\approx 2.5\mso$ might be unearthed by the vigorous winds during the planetary nebula phase. While the initial mass loss is predominantly spherically symmetric, once most of the hydrogen envelope is removed magnetic fields could appear at the surface of the compact central star of the planetary nebula (CSPN). This might initiate a highly asymmetric wind. The shape of NGC 6543 (the Cat's eye planetary nebula) is suggestive of such transition, with an outer spherical region and inner bipolar structure. The presence of discrete absorption components (DACs) in the CSPNs of NGC 6543 is also interesting \cite{Prinja_2012}, as in massive stars DACs have been connected to the presence of co-rotating interacting regions in the stellar wind, possibly caused by the presence of surface magnetic fields. 
