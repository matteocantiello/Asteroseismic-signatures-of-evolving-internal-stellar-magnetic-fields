\subsection{Magnetic Fields in Planetary Nebulae}
A large fraction of planetary nebulae have bipolar or other non-spherically symmetric shapes. 
One of the possible mechanisms for this lack of symmetry is the presence of a magnetic field in the central star \citep[see e.g.][]{Garc_a_Segura_1997,Chevalier_1994}. The magnetic fields generated by dynamos in the convective cores of main sequence stars above $\approx 2.5\mso$ might be unearthed by the vigorous mass loss during the planetary nebula phase. While the initial mass loss is predominantly spherically symmetric, once most of the hydrogen envelope is removed magnetic fields could appear at the surface of the compact central star of the planetary nebula (CSPN). This might initiate a highly asymmetric wind. The shape of NGC 6543 (the Cat's Eye planetary nebula) is suggestive of such a transition, with an outer spherical region and an inner bipolar structure. The presence of line profile variability in the CSPN of NGC 6543 is also interesting \citep{Prinja_2012}, as it is linked to the presence of co-rotating interacting regions in the stellar wind, possibly caused by the presence of surface magnetic fields. While this ``unearthed magnetic field'' scenario is appealing, observations of surface magnetic fields of CSPNs have mostly reported upper limits \citep{Jordan_2012,Leone_2014,Asensio_Ramos_2014}. Note that relatively weak  magnetic fields around AGB stars \citep{Leal_Ferreira_2013} and at the surface of AGB and post-AGB stars have been detected \citep{L_bre_2014,Sabin_2014}, although these fields are unlikely to be the same fields observed during the RGB using asteroseismology \citep{Fuller_2015}. %These fields might be generated by some dynamo process in the envelope during the AGB phase \citep[e.g.][]{Nordhaus_2007}.


%\subsection{Magnetic Fields in Planetary Nebulae}
%A large fraction of planetary nebulae have bipolar or other non-spherically symmetric shapes. 
%One of the possible mechanisms for this lack of symmetry is the presence of a magnetic field in the central star \citep[see e.g.][]{Garc_a_Segura_1997,Chevalier_1994}. The magnetic fields generated by dynamos in the convective cores of main sequence stars above $\approx 2.5\mso$ might be %unearthed by the vigorous mass loss during the planetary nebula phase. While the initial mass loss is predominantly spherically symmetric, once %most of the hydrogen envelope is removed magnetic fields could appear at the surface of the compact central star of the planetary nebula (CSPN). %This might initiate a highly asymmetric wind. The shape of NGC 6543 (the Cat's Eye planetary nebula) is suggestive of such a transition, with an %outer spherical region and an inner bipolar structure. The presence of line profile variability in the CSPN of NGC 6543 is also interesting %\citep{Prinja_2012}, as it is linked to the presence of co-rotating interacting regions in the stellar wind, possibly caused by the presence of %surface magnetic fields. While this ``unearthed magnetic field'' scenario is appealing, observations of surface magnetic fields of CSPNs have %mostly reported upper limits \citep{Jordan_2012,Leone_2014,Asensio_Ramos_2014}. Note that relatively weak  magnetic fields around AGB stars %\citep{Leal_Ferreira_2013} and at the surface of AGB and post-AGB stars have been detected \citep{L_bre_2014,Sabin_2014}, although these fields %are unlikely to be the same fields observed during the RGB using asteroseismology \citep{Fuller_2015}. %These fields might be generated by some %dynamo process in the envelope during the AGB phase \citep[e.g.][]{Nordhaus_2007}.
