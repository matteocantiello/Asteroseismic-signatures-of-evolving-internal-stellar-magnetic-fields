\section{Approximating the Core Magnetic Field Strength}
\label{Bcenap}

A rough estimate for the core magnetic field strength of a star with a convective MS core can be calculated as follows. The MS luminosity is efficiently carried by core convective motions, and implies an equipartition field strength of 
\begin{equation}
\label{eqn:Beq2}
B_{\rm eq} \simeq (4 \pi \rho)^{1/6} L^{1/3} r^{-2/3} \, ,
\end{equation}
where we have used $v_{\rm con} \simeq \big[L/(4 \pi \rho r^2)\big]^{1/3}$ to rewrite equation \ref{eqn:Beq}, and $L$ is the stellar luminosity. In general, this expression is a function of radius. At the edge of the convective core, we have
\begin{equation}
\label{eqn:rc}
r_{\rm c} = \bigg[ \frac{3 M_{\rm c}}{4 \pi {\bar \rho}_{\rm c}} \bigg]^{1/3} \, ,
\end{equation}
where $M_{\rm c}$ is the mass of the convective core and ${\bar \rho}_{\rm c}$ is its average density. Then we estimate within the convective core an approximate field strength of
\begin{equation}
\label{eqn:Beq3}
B_{\rm MS} \simeq 3^{-2/9} (4 \pi)^{7/18} \rho^{1/6} {\bar \rho}_{\rm c}^{2/9} L^{1/3} M_{\rm c}^{-2/9} \, .
\end{equation}
Within the convective core, the density does not change greatly. Since the expression above scales weakly with density, we can use the approximation $\rho \sim {\bar \rho}_{\rm c} \sim \rho_{\rm c}$ , where $\rho_{\rm c}$ is the central density. We also drop the numerical prefactor which is of order unity. Next, consider a sphere of density $\rho_{\rm c}$ near the center of the star. To conserve its mass, its radius evolves as $r_{\rm RG} = r_{\rm MS} (\rho_{\rm c,MS}/\rho_{\rm c,RG})^{1/3}$. Then assuming magnetic flux conservation after the MS as given in equation \ref{eqn:Brgb}, we arrive at 
\begin{equation}
\label{eqn:Bcen2}
B_{\rm RG} \sim L_{\rm MS}^{1/3} M_{\rm c,MS}^{-2/9} \rho_{\rm c,MS}^{-5/18} \rho_{\rm c,RG}^{2/3} \, .
\end{equation}

