\subsection{Magnetic Splitting}

Some red giants may contain magnetic fields in their core that are not strong enough to produce dipole mode suppression via the magnetic greenhouse effect. Instead, these stars may exhibit magnetically split oscillation modes. In the limit that $B \ll B_c$, the mode frequency perturbation is \citep{Unno_1989}
\begin{equation}
\label{eqn:dommag}
\frac{\delta \omega}{\omega} = \frac{1}{8 \pi \omega^2 I} \int d V | \delta {\bf B} |^2 \, .
\end{equation}
Here, $I$ is the mode inertia,  $\delta {\bf B}$ is the perturbation to the magnetic field,
\begin{equation}
\label{eqn:dB}
\delta {\bf B} = {\boldsymbol \nabla} \times \big( {\boldsymbol \xi} \times {\bf B} \big) \, ,
\end{equation}
and the integral is taken over the volume of the star. It is immediately evident that the amount of mode splitting is proportional to $B^2$, and is thus very small for $B \ll B_c$.

In Appendix A, we show that the magnetic mode splitting for dipole modes of frequency $\nu$ is approximately
\begin{equation}
\label{eqn:dfmag}
\delta \nu \sim \frac{\sqrt{\ell(\ell+1)}}{8 \pi} \Delta \nu_{\rm g} \int^{R_{\rm g}}_0 \frac{N}{\omega} \bigg(\frac{B_r}{B_c}\bigg)^2 \frac{dr}{r} \, .
\end{equation}
Here, $\Delta \nu_{\rm g}$ is the g mode frequency spacing between mixed modes, $B_r$ is the radial component of the field, $B_c$ is the critical field strength (equation \ref{eqn:Bc}), and the integral is taken over the g mode cavity. 
