\subsection{Magnetic Splitting}

Some red giants may contain magnetic fields in their core that are not strong enough to produce dipole mode suppression via the magnetic greenhouse effect. Instead, these stars may exhibit magnetically split oscillation modes. In the limit that $B \ll B_c$, the mode frequency perturbation is \citep{Unno_1989}
\begin{equation}
\label{eqn:dommag}
\frac{\delta \omega}{\omega} = \frac{1}{8 \pi \omega^2 I} \int d V | \delta {\bf B} |^2 \, .
\end{equation}
Here, $I$ is the mode inertia and  $\delta {\bf B}$ is the perturbation to the magnetic field,
\begin{equation}
\label{eqn:dB}
\delta {\bf B} = {\bf \nabla} \times \big( \boldmath \xi} \times {\bf B} \big) \, .
\end{equation}