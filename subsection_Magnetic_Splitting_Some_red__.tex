\subsection{Magnetic Splitting}

Some red giants may contain magnetic fields in their core that are not strong enough to produce dipole mode suppression via the magnetic greenhouse effect. Instead, these stars may exhibit magnetically split oscillation modes. In the limit that $B \ll B_c$, the mode frequency perturbation is \citep{Unno_1989}
\begin{equation}
\label{eqn:dommag}
\frac{\delta \omega_{\rm M}}{\omega} = \frac{1}{8 \pi \omega^2 I} \int d V | \delta {\bf B} |^2 \, .
\end{equation}
Here, $I$ is the mode inertia,  $\delta {\bf B}$ is the perturbation to the magnetic field,
\begin{equation}
\label{eqn:dB}
\delta {\bf B} = {\boldsymbol \nabla} \times \big( {\boldsymbol \xi} \times {\bf B} \big) \, ,
\end{equation}
and the integral is taken over the volume of the star. It is immediately evident that the amount of mode splitting is proportional to $B^2$, and is thus very small for $B \ll B_c$.

In Appendix A, we show that the magnetic mode splitting for dipole modes of frequency $\nu$ is approximately
\begin{equation}
\label{eqn:dfmag}
\delta \nu_{\rm M} \sim \frac{\sqrt{\ell(\ell+1)}}{8 \pi} \Delta \nu_{\rm g} \int^{R_{\rm g}}_0 \frac{N}{\omega} \bigg(\frac{B_r}{B_c}\bigg)^2 \frac{dr}{r} \, .
\end{equation}
Here, $\Delta \nu_{\rm g}$ is the g mode frequency spacing between mixed modes, $B_r$ is the radial component of the field, $B_c$ is the critical field strength (equation \ref{eqn:Bc}), and the integral is taken over the g mode cavity

To estimate the level of magnetic splitting expected, we evaluate equation \ref{eqn:dfmag} for the red giant model shown in Figure S1 of \cite{Fuller_2015}, but with a field strength weaker by a factor of 10, such that $B_r \simeq B_c$ at the H-burning shell and $B_r \ll B_c$ everywhere else. This is roughly the maximum field strength that could exist without dipole mode suppression, and this particular model has a central field strength of $B_r \approx 7 \times 10^5 \, {\rm G}$.  In this case, we find $\delta \nu_{\rm M} \sim 2 \, \mu {\rm Hz}$, whereas $\Delta \nu_{\rm g} \sim 1 \, \mu{\rm Hz}$. Magnetic splitting may therefore be comparable to the g mode period spacing in RGB stars on the verge of dipole mode suppression. The magnetic splitting may also be comparable to rotational splitting. In these stars (of which ``Droopy", KIC 8561221 \cite{Garcia_2014} may be an example), magnetic splitting may complicate the interpretation of g mode period spacing ant rotational splitting. We abstain from a more thorough analysis of mode splitting in these stars, as it depends on both field geometry and core rotation rate. However, in most ``normal" stars with slighltly weaker fields such that $B_r \ll B_c$ everywhere, we expect $\delta \nu_{\rm M}$ to be much smaller than rotational splitting and is likely undetectable. 




