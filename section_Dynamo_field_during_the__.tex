\section{Dynamo field during the MS}
\subsection{Operation of Main-Sequence Dynamo}
The main sequence of stars more massive than about 1.1$\mso$ is characterized by the presence of a convective core.
This is due to central temperatures high enough to activate the CNO cycle, whose nuclear reaction rates depend more steeply on temperature than the p-p chain ($\epsilon_{\rm pp}\propto \rho T^{4}$, while $\epsilon_{\rm CNO}\propto \rho T^{18}$ ).

Due to the highly conductive nature of stellar plasmas, in the presence of rotation
these cores are expected to host a contemporary dynamo. Dynamo action converts some of the kinetic energy 
of the convective motions into magnetic energy, with magnetic fields sustained against dissipation  \citep[see e.g.,][]{Brandenburg_2005}. Magneto hydrodynamics calculations of the central regions of a  $2\mso$ A-type star rotating with 
1 and 4 times the solar mean angular velocity (rotation periods of 28 and 7 days) show dynamo action 
with magnetic fields reaching a considerable fraction of the equipartition energy \citep{Brun_2005}.
Large scale magnetic fields produced in the core are not expected to diffuse outside and reach the surface, due to the fact that for these stars the Ohmic diffusion timescale ($\tau_{\rm Ohm}\sim \rso^2/\eta$, with $\eta$ the magnetic diffusivity) is longer than the stellar lifetime. \citet{MacGregor_2003} discuss the possibility that magnetic buoyancy instabilities can bring small, magnetized fibrils at the stellar surface. However the inclusion of realistic compositional gradients seems to disfavor this scenario, increasing considerably the timescales of magnetic buoyancy \citep{MacDonald_2004}.  




- Dynamo Action

- Simulations (Brun)

- Role of Rossby Numbers

- Kraft Break
  
- Typical Ro and Beq   
  
  
  
  
  
  