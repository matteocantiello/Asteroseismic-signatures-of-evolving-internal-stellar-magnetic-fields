\section{Conclusions}
\label{conclusion}

The main goal of this paper is to examine the implications of asteroseismic detections of strong internal magnetic fields in low-mass red giant stars via suppressed dipole oscillation modes (Fuller et al. 2015, Stello et al. 2015), and to generate predictions, extrapolations, and guidelines for future work. 

First, we examined the visibility of suppressed oscillation modes in stars with strong core magnetic fields. Quadrupole modes are predicted to exhibit small (but detectable) suppression which can be used to check the validity of the existing theory and observational techniques. Dipole mode suppression should be easily detectable in clump stars in addition to stars ascending the RGB. For stars with $M \lesssim 2 \, M_\odot$, clump stars are expected to show lower dipole mode visibility than ascending RGB stars at the same $\nu_{\rm max}$, and therefore clump and RGB stars might be able to be distinguished from one another.

Next, we investigated the evolution of magnetic fields created by MS convective core dynamos during the post-MS evolution. Magnetic diffusion timescales within the red giant core are generally longer than stellar evolution time scales. Therefore, the fields will be frozen into the mass coordinates at which they form (i.e., mass coordinates within the MS convective core), and will not migrate outward during red giant evolution. In stars with $M \lesssim 1.5 \, M_\odot$, the H-burning shell lies above the extent of the MS convective core during evolution on the lower RGB. Therefore, strong fields are not expected to be present at the location of the H-burning shell in these stars, which may account for the low incidence of dipole suppression in ascending RGB stars with $M \leq 1.5 \, M_\odot$ (Stello et al. 2015). The opposite is true for stars with $M \gtrsim 1.5 \, M_\odot$, accounting for their much larger dipole mode suppression rate.

We then examined the possibility of magnetic mode suppression for clump stars. Fields strengths in the range of $2-20 \times 10^4 \, {\rm G}$ at the H-burning shell are required for mode suppression in clump stars, or fields a few times larger in the radiative He above the convective He-burning core. However, it is not clear whether such fields will persist into the clump phase, as it is possible that they are erased during post-MS convective phases such as the deeply penetrating convective envelope and the convective shells that arise during He flashes. For stars with $M \lesssim 2.1 \, M_\odot$, we find that all mass coordinates within the radiative core on the clump were convective during the RGB, which is not true for secondary clump stars with $M \gtrsim 2.1 \, M_\odot$. Therefore, the observed incidence of dipole mode suppression in clump stars will have telling consequences for magnetic field evolution within stars. A large percentage of low-mass clump stars with suppressed modes would indicate that fields are robust and able to withstand convective phases, while suppressed modes only in secondary clump stars would indicate that large scale fields are generally destroyed by convective phases.

Finally, we discussed implications for the magnetic fields observed in white dwarfs (WDs) and neutron stars (NSs). We suggest that strong fields are more common at the surfaces of massive WDs because they had more massive progenitors. Only the massive progenitors ($M \gtrsim 3 \, M_\odot$) had MS convective cores which encompassed the entire mass of the WD remnant, and hence only these stars could have generated strong MS fields capable of being observed at the surface of the WD. In NS progenitors, the MS convective core always extends to mass coordinates larger than the NS mass, and equipartition fields entail NS field strengths of $\sim \! 10^{15} \, {\rm G}$ if magnetic flux is conserved after the MS. Therefore, MS dynamos may be capable of creating the fields observed in magnetars. However, since most NSs are not magnetars, the MS fields must be unlikely to survive into the NS phase, perhaps indicating they are destroyed by subsequent convective core/shell burning phases. The origin of magnetar fields thus has two possibilities: either some post-MS process destroys most (but not all) MS dynamo-generated fields, or some post-MS process (e.g., subsequent convective burning phases or a proto-NS dynamo) is able to generate strong fields that survive into the NS phase for a small fraction of NS progenitors.
  
  
  
  
  
  
  
  
  
  