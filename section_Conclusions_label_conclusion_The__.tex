\section{Conclusions}
\label{conclusion}

The main goal of this paper is to examine the implications of asteroseismic detections of strong internal magnetic fields in low-mass red giant stars (Fuller et al. 2015, Stello et al. 2015), and to generate predictions and guidelines for future work. 

First, we examined the visibility of oscillation modes in stars with strong core magnetic fields. Quadrupole modes are predicted to exhibit small (but detectable) suppression which can be used to check the validity of the existing theory and observational techniques. Magnetic suppression should be easily detectable in clump stars in addition to stars ascending the RGB. For stars with $M \lesssim 2 \, M_\odot$, clump stars are expected to show lower dipole mode visibility than ascending RGB stars at the same $\nu_{\rm max}$, and therefore clump and RGB stars might be able to be distinguished from one another.

Next, we investigated the evolution of magnetic fields created by main sequence dynamos during the post-main sequence evolution. Magnetic diffusion timescales within the red giant core are generally longer than stellar evolution time scales. Therefore, the fields will be frozen into the mass coordinates at which they form (i.e., mass coordinates within the main sequence convective core), and will not migrate outward during red giant evolution.


  
  