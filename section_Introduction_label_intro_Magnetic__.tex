\section{Introduction}
\label{intro}
Magnetic fields are ubiquitous in astrophysics. From  galaxy clusters to strongly magnetized neutron stars (Magnetars) their amplitude spans more than 20 orders of magnitudes \citep{Brandenburg_2005}. Using the Zeeman effect, stellar magnetic fields have been measured at the surface of stars other than the Sun for more than 60 years \citep{Babcock_1947}. 
Stars with convective envelopes show the presence of surface magnetic fields that are believed to be produced by contemporary dynamo action. A fraction of stars with radiative envelopes also have strong magnetic fields, that could be generated or inherited during the star formation process or an earlier evolutionary phase (fossil fields). This fraction appear to be 5-10\% for main sequence stars of spectral class A and OB \citep{2012ASPC..464..405W}. On the other hand small scale amplitude fields could be ubiquitous at the surface of A \ and OB stars \citep{Cantiello_2011}, and while identified in some A type stars like Vega ans Sirius, their detection in more massive stars is challenging \cite{2013A&A...554A..93K}.
The presence of  stellar magnetic fields is also inferred indirectly by phenomena associated with their surface manifestations, like stellar activity, x-ray emission, angular momentum loss and structures in the stellar wind. 



\begin{itemize}
\item Stellar Magnetic fields
\item Red giant Asteroseismology
\item Clump stars
\item Depressed Modes
\end{itemize}


  