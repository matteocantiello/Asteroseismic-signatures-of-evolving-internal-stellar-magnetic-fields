\section{Introduction}
\label{intro}
Magnetic fields are ubiquitous in astrophysics. From  galaxy clusters to strongly magnetized neutron stars (magnetars) their amplitude spans more than 20 orders of magnitudes \citep{Brandenburg_2005}. Using the Zeeman effect, stellar magnetic fields have been measured at the surface of stars
%other than the Sun
for more than 60 years \citep{Babcock_1947,Landstreet_1992,Donati_2009}. Stars with convective envelopes show the presence of surface magnetic fields that are believed to be produced by contemporary dynamo action. A fraction of stars with radiative envelopes also have strong (B $\sim$ kG) large scale magnetic fields that are likely generated or inherited during the star formation process (fossil fields). This fraction appears to be 5-10\% for main sequence stars of spectral class A \citep[e.g.,][]{Auri_re_2004} and OB \citep{2012ASPC..464..405W}.

On the other hand, weak (B $\lessim$ 100 G) and/or small scale fields could be much more common at the surface of A and OB stars \citep{Cantiello_2011,Braithwaite_2012}. While such fields have been identified in some A type stars like Vega \citep{Ligni_res_2009}, their detection in more massive stars is challenging \citep{2013A&A...554A..93K}. The presence of stellar magnetic fields is also inferred indirectly by phenomena associated with their surface manifestations, like stellar activity, x-ray emission, angular momentum loss and structures in the stellar wind. Until recently, evidence for stellar magnetism has been limited to surface fields, with compact remnants providing clues about the level of internal magnetization of their progenitor stars.

Thanks to a new asteroseismic technique, strong internal magnetic fields can now be detected in red giant stars \citep{Fuller_2015}. 
[TBC]
%\begin{itemize}
%\item Stellar Magnetic fields
%\item Red giant Asteroseismology
%\item Clump stars
%\item Depressed Modes
%\end{itemize}


  