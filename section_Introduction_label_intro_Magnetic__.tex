\section{Introduction}
\label{intro}
Magnetic fields are ubiquitous in astrophysics. From  galaxy clusters to strongly magnetized neutron stars (magnetars) their amplitude spans more than 20 orders of magnitudes \citep{Brandenburg_2005}. Using the Zeeman effect, stellar magnetic fields have been measured at the surface of stars
%other than the Sun
for more than 60 years \citep{Babcock_1947,Landstreet_1992,Donati_2009}. Stars with convective envelopes show the presence of surface magnetic fields that are believed to be produced by contemporary dynamo action. A small fraction of stars with radiative envelopes also have strong ($B \! \sim$kG) large scale magnetic fields that are likely generated or inherited during the star formation process (fossil fields). This fraction appears to be 5-10\% for main sequence stars of spectral class A \citep[e.g.,][]{Auriere2004} and OB \citep{2012ASPC..464..405W}.

On the other hand, weak ($B \! \lesssim$100 G) and/or small scale fields could be much more common at the surface of A and OB stars \citep{Cantiello_2011,Braithwaite_2012}. While such fields have been identified in some A type stars like Vega \citep{Lignieres2009}, their detection in more massive stars is challenging \citep{2013A&A...554A..93K}. The presence of stellar magnetic fields is also inferred indirectly by phenomena associated with their surface manifestations, like stellar activity, x-ray emission, angular momentum loss and structures in the stellar wind. Until recently, evidence for stellar magnetism has been limited to surface fields, with only compact remnants providing clues about the level of internal magnetization of their progenitor stars.

Thanks to a new asteroseismic technique, strong internal magnetic fields can now be detected in red giant stars \citep{Fuller_2015}. This is done using mixed modes, oscillations that live as both p-modes in the acoustic cavity and g-modes in the stellar core \cite{Dupret_2009}. Mixed modes have been used successfully  to determine the evolutionary status of red giant stars \citep{Bedding_2011}, and to measure their internal rotation profile \citep{Beck_2011}.  
The excitation of these pulsation modes is provided by turbulent convection in the red giant envelope, with part of the wave energy leaking through the evanescent region into the g-mode cavity. The presence of a strong magnetic field in the core is able to alter the propagation of the gravity waves, trapping  mode energy in the core and effectively decreasing its visibility \citep[magnetic greenhouse effect,][]{Fuller_2015}. Red giants with strong internal magnetic fields can then be identified by the presence of depressed oscillation modes in their oscillation spectrum. The amplitude of oscillation modes in RGB with strongly magnetized cores depends on the amount of coupling between the p-mode and g-mode cavities. This is regulated by the extension of the evanescent region, which at fixed evolutionary stage depends primarily on the harmonic degree of the mode. Dipolar modes have maximum coupling and therefore do show the largest amplitude depression in the presence of strong core magnetic fields. Theoretical predictions from \citet{Fuller_2015} have been verified by \citet{Stello_2016}, who applied the theory to a large sample of {\it Kepler} red giants and found that the amplitudes of depressed dipole oscillation modes are consistent with total wave energy loss in the core, although it remains possible that some small fraction of the wave energy returns to the envelope.

In Sec.~\ref{visibility} we extend the work of \citet{Fuller_2015} and show theoretical predictions for the visibility of depressed dipolar ($\ell=1$) and quadrupolar ($\ell=2$) oscillation modes during the red giant  (H-burning) and  the red clump (core He-burning) phases. 


%\begin{itemize}
%\item Stellar Magnetic fields
%\item Red giant Asteroseismology
%\item Clump stars
%\item Depressed Modes
%\end{itemize}


  