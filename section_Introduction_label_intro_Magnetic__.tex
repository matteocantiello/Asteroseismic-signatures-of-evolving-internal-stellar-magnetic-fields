\section{Introduction}\label{intro}
Magnetic fields are ubiquitous in astrophysics. From  galaxy clusters to strongly magnetized neutron stars (magnetars) their amplitude spans more than 20 orders of magnitudes \citep{Brandenburg_2005}. Using the Zeeman effect, stellar magnetic fields have been measured at the surface of stars
%other than the Sun
for more than 60 years \citep{Babcock_1947,Landstreet_1992,Donati_2009}. Stars with convective envelopes show the presence of surface magnetic fields that are believed to be produced by contemporary dynamo action. A small fraction of stars with radiative envelopes also have strong ($B \! \sim \! {\rm kG}$) large scale magnetic fields that are likely generated or inherited during the star formation process (fossil fields). This fraction appears to be 5-10\% for main sequence stars of spectral class A \citep[e.g.,][]{Auriere2004} and OB \citep{2012ASPC..464..405W}.

On the other hand, weak ($B \! \lesssim \! 100 \, {\rm G}$) and/or small scale fields could be much more common at the surface of A and OB stars \citep{Cantiello_2011,Braithwaite_2012}. While such fields have been identified in some A type stars like Vega \citep{Lignieres2009}, their detection in more massive stars is challenging \citep{2013A&A...554A..93K}. The presence of stellar magnetic fields is also inferred indirectly by phenomena associated with their surface manifestations, like stellar activity, x-ray emission, angular momentum loss and structures in the stellar wind. Until recently, evidence for stellar magnetism has been limited to surface fields, with only compact remnants providing clues about the level of internal magnetization of their progenitor stars.

Thanks to a new asteroseismic technique, strong internal magnetic fields can now be detected in red giant stars \citep{Fuller_2015}. This is done using mixed modes, oscillations that behave as pressure waves (p modes) in the acoustic cavity and gravity waves (g modes) in the stellar core \cite{Dupret_2009}. Mixed modes have been used successfully to determine the evolutionary status of red giant stars \citep{Bedding_2011}, and to measure their internal rotation profile \citep{Beck_2011}.  The excitation of these pulsation modes is provided by turbulent convection in the red giant envelope, with part of the wave energy leaking through the evanescent region into the g mode cavity. The presence of a strong magnetic field in the core is able to alter the propagation of the gravity waves, trapping  mode energy in the core and effectively decreasing its visibility \citep[magnetic greenhouse effect,]{Fuller_2015}. Theoretical predictions from \citet{Fuller_2015} have been verified by \citet{Stello_2016}, who applied the theory to a large sample of {\it Kepler} ascending red giant branch (RGB) stars and found that the amplitudes of depressed dipole oscillation modes are consistent with nearly total wave energy loss in the core.
%although it remains possible that some small fraction of the wave energy returns to the envelope.


Red giants with strong internal magnetic fields can thus be identified by the presence of suppressed oscillation modes in their oscillation spectra. These stars allow for a calculation of the minimum magnetic field $B_{c,{\rm min}}$ that must exist in the core. The amplitude of the suppressed oscillation modes depends on the amount of coupling between the p mode and g mode cavities, which is regulated by the extent of the evanescent region and depends on the angular degree $\ell$ of the mode. Dipolar ($\ell=1$) modes have maximum coupling and therefore show the largest amplitude depression in the presence of strong core magnetic fields. \cite{Stello_2016} showed that strong magnetic fields are present in roughly 50\% of RGB stars with $M \gtrsim 1.5 \, M_\odot$, but are very rare in stars with $M \lesssim 1.1 \, M_\odot$. They interpreted this dichotomy as an effect of main sequence core dynamo-generated magnetic fields, which are generated in the convective cores of $M \gtrsim 1.1 \, M_\odot$ stars.

In this paper, we extend the analyses of \cite{Fuller_2015} and \cite{Stello_2016}, providing additional analysis of the generation, evolution, and detectability of internal stellar magnetic fields. We find that quadrupole ($\ell=2$) modes should exhibit detectable suppression in red giants with magnetic cores, which could be used to validate the theory of \cite{Fuller_2015}. Moreover, we find that clump stars with magnetic cores will exhibit highly suppressed dipole modes, and should be easily detectable in {\it Kepler} data. Moderate field strengths ($B \gtrsim 10^4$-$10^5 \, {\rm G}$) are sufficient for dipole mode suppression in clump stars, provided those fields exist within the stably stratified shell around the convective He-burning core.

Next, we examine the generation and evolution of magnetic fields in stellar cores. Although strong ($B \! > \! 10^5 \, {\rm G}$) magnetic fields are likely a common outcome of core dynamos, the fields are confined within the mass coordinates of the convective region, which can help explain the rising incidence of magnetic fields in stars with $1.1 \, M_\odot \! \lesssim \! M \! \lesssim \! 1.5 \, M_\odot$ due to the larger convective core of the higher mass stars. We find it difficult to predict whether the core magnetic fields will survive through core He-burning and into the white dwarf stage of evolution, as magnetic fields could be altered, amplified, or erased by convective He-burning. However, we outline how observations of suppressed dipole modes in clump stars can be used to distinguish the possibilities and provide a clear picture of internal magnetic field evolution throughout stellar evolution. Finally, we examine connections with magnetic fields in compact remnants and other types of stellar pulsators. We show that core dynamo-generated fields could be responsible for the strong magnetic fields at the surfaces of some white dwarfs and neutron stars. Core fields may also affect or suppress pulsations in some slowly pulsating B type stars, $\gamma$-Doradus stars, and subdwarf-B stars. 

In Section \ref{visibility}, we calculate theoretical predictions for the visibility of suppressed dipolar and quadrupolar oscillation modes during the red giant (H-burning) and the red clump (core He-burning) phases. Section \ref{dynamo} examines the theory of magnetic field generation and evolution, and how this relates to observations of RGB stars, while Section \ref{clump} extends this analysis to clump stars. We discuss connections with magnetic fields in compact remnants in Section \ref{remnants} and other types of pulsators in Section \ref{pulsators}, before concluding in Section \ref{conclusion}. 


%\begin{itemize}
%\item Stellar Magnetic fields
%\item Red giant Asteroseismology
%\item Clump stars
%\item Depressed Modes
%\end{itemize}


  