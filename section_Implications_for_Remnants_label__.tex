\section{Implications for Remnants}
\label{remnants}

\subsection{Magnetic Fields in White Dwarfs}

Many white dwarfs (WDs) exhibit strong ($B \gtrsim 3 \, {\rm MG}$) surface magnetic fields. The exact fraction of WDs which have strong fields is debated but is of order 10\% \citep{Hollands_2015}. Moreover, there is a dearth of magnetic WDs with field strengths below $10 \, {\rm MG}$, and the magnetic WDs are systematically more massive \citep{Ferrario_2015B}.

Our work suggests that some of the strong fields observed at WD surfaces could be the remnants of MS core-dynamo-generated fields. Equipartition fields of $\sim \! 2 \times 10^5 \, {\rm G}$ generated by a core dynamo would evolve into fields of $\sim 2 \times 10^7 \, {\rm G}$ if their flux is conserved from the MS to the WD phase. Thus, the field strengths inferred for core dynamo-generated fields are squarely within the observed distribution of WD surface fields \citep{Ferrario_2015A,Ferrario_2015B}.

Figure \ref{fig:DipoleHist} shows that the core convective region is restricted to mass coordinates below $\approx 0.5 \, M_\odot$ for stars of $M \lesssim 2.5 M_\odot$. Using the MS-WD mass relation of \cite{Renedo_2010}, these low-mass stars account for the majority of WDs, whose mass distribution peaks near $0.6 \, M_\odot$ \citep{Rebassa-Mansergas_2015}. Since the ohmic diffusion time is very long in WDs \citep{Ferrario_2015B}, we expect that core dynamo-generated fields are unlikely to be visible at the surface of most WDs. 

However, in stars of $M \gtrsim 3 \, M_\odot$, the convective core extends to mass coordinates larger than $0.6 \, M_\odot$.  In these stars, the entire mass of the WD descendant lies within the mass coordinate occupied by the MS convective core. Therefore, strongly magnetized regions may extend all the way to the WD surface where they can be observed. Stars over $M \gtrsim 3 \, M_\odot$ produce WDs of $M_{\rm WD} \gtrsim 0.7 \, M_\odot$, similar to the typical masses of magnetic WDs \citep{Ferrario_2015B}. Thus, some magnetic WDs may be magnetized due to MS convective core dynamos, which could partially explain why magnetic WDs are more massive on average. However, it remains possible that magnetic WDs are the descendants of magnetic Ap stars, or that they are formed through WD mergers. 

The interesting corollary to this discussion is that strong magnetic fields may exist within the interiors of many WDs even though the fields are not visible at the surface. Indeed, based on the MS-WD mass relation, most WDs originate from progenitors of $M \gtrsim 1.5 \, M_\odot$ in which core dynamos operate and produce strong magnetic fields in over 50\% of RGB stellar cores \citep{Stello_2016}. We therefore speculate that many (perhaps the majority of) WDs could contain strong ($B \gtrsim 10^6 \, {\rm G}$) magnetic fields which are confined within the stellar interior and not detectable at the surface even as they cool \citep[$t_{\rm Ohm} \sim 10^{11}$ yrs in WDs interiors,][]{Cumming_2003}. These magnetic fields may have very important implications for WD evolution, and for the outcome of WD mergers. 


\subsubsection{Helium-Core White Dwarfs}

Helium-core white dwarfs (He WDs) typically have masses in the range $0.15 \, M_\odot \lesssim M \lesssim 0.4 \, M_\odot$ and are formed when a companion star strips the hydrogen envelope of the He WD progenitor as it ascends the RGB. He WDs are essentially the naked cores of the RGB stars analyzed in \cite{Fuller_2015} and \cite{Stello_2016}. Unless internal magnetic fields are somehow destroyed by envelope mass loss, we expect some He WDs to exhibit surface fields of $B \gtrsim 10^5 \, {\rm G}$. As far as we are aware, strong magnetic fields have not yet been observed at the surfaces of any He WDs, even though they may be detectable.

Predicting the fraction of He WDs that will exhibit strong surface fields is not straightforward, as it depends both on the progenitor mass and the He WD mass. For instance, the $1.75 \, M_\odot$ model shown in Figure \ref{fig:DipoleHist} has an MS convective core that extends to a mass coordinate of $\approx 0.24 \, M_\odot$. Therefore, its He WD descendant may only exhibit strong surface fields if its mass is $M_{\rm WD} \lesssim 0.24 \, M_\odot$, otherwise the fields may remain buried. We encourage searches for magnetic fields in He WDs, as their detection would allow further characterization of the strong fields inferred to exist within red giant cores.



\subsection{Magnetic Fields in Neutron Stars}

Since the observations of \cite{Stello_2016} show that core-dynamo-generated magnetic fields frequently survive well into RGB evolution in low-mass stars, it is possible that these fields are also long-lived in massive stars that spawn neutron stars upon their death. We find typical equipartition field strengths of $B \sim 10^6 \, {\rm G}$ in the MS convective core of $M \sim 12 \, M_\odot$ neutron star progenitors. Flux conservation of the field within the inner $1.4 \, M_\odot$ (which has a radius of $\sim \! 0.5 \, R_\odot$ on the MS) to a neutron star radius of $12 \, {\rm km}$ would lead to neutron star surface field strengths of $\sim \! 10^{15} \, {\rm G}$, i.e., magnetar field strengths.  

The magnetar birth rate is highly uncertain, with plausible estimates in the range of \citep{keane_1998,mereghetti_2015} $5-50\%$ of the neutron star birth rate. It is therefore possible that stable magnetic fields in the cores massive stars are just as common as in low-mass stars, leading to magnetar birth rates of order $50 \%$ the neutron star birth rate.  
%One possibility is that the core-dynamo-generated fields are less likely to settle into a stable configuration in massive stars compared to their lower mass counterparts.
If magnetar birth rates turn out to be smaller, it may indicate that post-MS core convective phases (He, C, Ne, O, or Si burning) destroy MS core dynamo-generated fields and prevent magnetar formation, as we have hypothesized to happen in low-mass stars. 
%This is certainly possible, especially if angular momentum transport is efficient in massive stars (as it appears to be in low-mass stars) such that their cores are slowly rotating in post-MS phases of evolution and are not able to sustain a rotation-driven dynamo.
An absence of dipole mode suppression in low-mass clump stars would support this hypothesis. It also remains possible that most magnetars are the descendants of magnetic OB stars, or that their fields are generated during a proto-neutron star dynamo \cite{1992ApJ...392L...9D}.



  
  
  
  
  
  
  
  
  
  
  