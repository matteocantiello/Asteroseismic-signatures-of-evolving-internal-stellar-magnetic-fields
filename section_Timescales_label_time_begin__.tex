\section{Timescales}
\label{time}
Here we discuss the relevant timescales affecting the evolution of internal magnetic fields and their ability to 
suppress oscillation modes through a magnetic greenhouse effect (Fuller et al. 2015).
We will assume that at the end of the main sequence a magnetic field is present below the maximum lagrangian mass occupied, at any time, by the convective core.

The first important timescale is the Ohmic timescale $\t_{\rm Ohm}$, the time it takes to a stable magnetic field in a radiative region to diffuse a pressure scale height. This timescale is usually quite long, due to the large values of conductivity of stellar plasma. Figure~\ref{fig:timescales} shows that in the core of a $1.5\mso$ evolving from the main sequence to the RGB,  $\t_{\rm Ohm}$ is quite long, and varies between $10^8-10^12$ yrs. Note that  $\t_{\rm Ohm}$ does not depend on the amplitude or geometry of the magnetic field, but only on the local value of the magnetic diffusivity. The magnetic diffusivity is the inverse of the electrical conductivity,  which in these stars has to be calculated carefully as certain regions are partially/fully degenerate. We adopt

The other important timescale that determines the relative location of magnetic fields in  RGB stars is the H-shell burning timescale. As a star climbs the RGB, the ashes of H-shell burning increase the size of the He core. If the timescale of this process ($t_{\rm Acc}$ accretion timescale)  is shorter than the Ohmic diffusion, then the magnetic field can be buried below the He raining from the H-shell. Figure~\ref{fig:timescales} shows that in a $1.5\mso$ this is always the case, meaning that 
we can safely assume that magnetic fields present in the stellar core at the end of the main sequence are frozen in their lagrangian mass coordinate. If the He core grows substantially during the subgiant / early RGB phases, magnetic fields can be efficiently buried below the H-shell (see e.g. the $1.25\mso$ model in Fig.~\label{fig:DipoleHist}), the location where the waves are most sensitive to the magnetic greenhouse effect. 

