\subsection{Red Clump}
\label{clump}
  

In red giants, mixed modes are generally observable when the evanescent region separating the core and envelope is quite narrow, which occurs for RGB stars below the luminosity bump. As stars evolve further up the RGB and expand, the evanescent region thickens, and mixed modes become undetectable. The mixed modes become visible again after the star has contracted and settled onto the clump for the core He burning phase. 

The degree of mode suppression due to the magnetic greenhouse effect in red giants behaves in a very similar fashion to mixed mode visibility, because both effects require wave energy to tunnel into the core. Therefore, dipole modes will have strongly suppressed visibilities for clump stars, provided that sufficiently strong magnetic fields exist in the radiative cores of clump stars. Figure \ref{fig:visibility} shows the predicted power of dipole and quadrupole modes for clump stars of various masses. We predict typical suppressed powers of $0.1 \lesssim (V_{\rm sup}/V_{\rm norm})^2 \lesssim 0.4$ for suppressed dipole modes in clump stars, with more massive stars exhibiting slightly lower mode visibility. Quadrupole modes are expected to have $0.4 \lesssim (V_{\rm sup}/V_{\rm norm})^2 \lesssim 0.8$. 

For stars with $M \lesssim 2.0 \, M_\odot$, dipole modes will have smaller visibilities in clump stars than they will for RGB stars of the same mass and $\nu_{\rm max}$. Therefore, low-mass clump stars with suppressed modes may stick out by virtue of exhibiting exceptionally low dipole mode visibilities for stars with $\nu_{\rm max} \sim 20-50 \, \mu {\rm Hz}$. These stars may be visible as the group of stars with $V^2 \sim 0.4$ and $\nu_{\rm max} \sim 40 \, \mu {\rm Hz}$ in Figure 2 of Fuller et al. 2015. We caution that this region of $\nu_{\rm max}$ and visibility space is also inhabited by more massive stars ($M \gtrsim 2 \, M_\odot$) with suppressed modes that are ascending the RGB, although these two populations can be distinguished by their different masses. In larger samples, the existence of absence of a large population of stars with $V^2 \sim 0.2-0.4$ and $\nu_{\rm max} \sim 20-50 \, \mu {\rm Hz}$ will indicate whether magnetic mode suppression commonly operates within clump stars of $M \lesssim 2.0 \, M_\odot$. Massive clump stars ($M \gtrsim 2 \, M_\odot$) with suppressed modes will have similar mode visibilities to their RGB counterparts at the same $\nu_{\rm max}$, and distinguishing RGB from clump stars may be difficult.
  
  
  
  
  
  
  
  