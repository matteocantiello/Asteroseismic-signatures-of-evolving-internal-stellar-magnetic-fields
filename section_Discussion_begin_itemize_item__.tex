\section{Discussion}
\begin{itemize}
\item Rotational suppression
\item J-transport
\item Magntized remnants
\item OB stars
\item Type Ia SN (all WD have buried internal fields?)
\item Fossil Fields
\end{itemize}


\subsection{Angular Momentum Transport}

Several asteroseismic studies \citep{Beck_2012,Mosser_2012,Deheuvels_2014,Deheuvels_2015} have measured the core rotation rates of RGB/clump stars. The relatively slow core rotation rates indicate strong angular momentum transport mechanisms are at work \citep{Cantiello_2014}, coupling the radiative cores with the convective envelope. The strong magnetic fields frequently found in the cores of RGB stars (Stello et al. 2015) may play an important role in this process. Our work suggests that strong magnetic fields restricted to mass coordinates of RGB stars that were convective on the MS. For stars of $M \lesssim 1.5 \, M_\odot$, the strong fields are restricted to the He core, and cannot directly couple the core with the envelope. Since the majority of the sample of \cite{Mosser_2012} has $M \lesssim 1.5 \, M_\odot$, core-dynamo-generated fields cannot solely account for slow core rotation.

More importantly, the sample of stars with measured core rotation rates are mutually exclusive from the sample of stars with strong core magnetic fields. The reason is that mixed dipole modes are used to measure the core rotation rates, but these modes are highly suppressed/absent in stars with magnetic cores. In order for large-scale magnetic fields to account for the measured core rotation rates, they must extend from the inner helium core to the outer radiative core, and they must be weak enough that they do not suppress dipole oscillation modes. Unfortunately, neither the current study nor the measurements of Stello et al. 2015 can provide useful constraints on the existence of such fields. 


\subsection{Magnetic Fields in White Dwarfs}

Many white dwarfs (WDs) exhibit strong ($B \gtrsim 3 \, {\rm MG}$) surface magnetic fields. The exact fraction of WDs which have strong fields is debated but is of oder 10\% \citep{Hollands_2015}. Moreover, there is a dearth of magnetic WDs with field strengths below $10 \, {\rm MG}$, and the magnetic WDs are systematically more massive \citep{Ferrario_2015B}.

Our work suggests that some of the strong fields observed at WD surfaces could be the remnants of MS core-dynamo-generated fields. Equipartition fields of $\sim \! 2 \times 10^5 \, {\rm G}$ generated by a core dynamo would evolve into fields of $\sim 2 \times 10^7 \, {\rm G}$ if their flux is conserved from the MS to the WD phase. Thus, the field strengths inferred for core-dynamo-generated fields are squarely within the observed distribution of WD surface fields \citep{Ferrario_2015B}.

Figure \ref{fig:DipoleHist} shows that the core convective region is restricted to mass coordinates below $\approx 0.5 \, M_\odot$ for stars of $M \lesssim 2.5 M_\odot$. Using the MS-WD mass relation of \cite{Renedo_2010}, these low-mass stars account for the majority of WDs, whose mass distribution peaks near $0.6 \, M_\odot$ (***REF***). Since the ohmic diffusion time is very long in WDs \citep{Ferrario_2015B}, we expect that core-dynamo-generated fields are unlikely to be visible at the surface of most WDs. 

However, in stars of $M \gtrsim 3 \, M_\odot$, the convective core extends to mass coordinates larger than $0.6 \, M_\odot$. Stars over $M \gtrsim 3 \, M_\odot$ produce WDs of $M_{\rm WD} \gtrsim 0.7 \, M_\odot$, similar to the typical masses of magnetic WDs \citep{Ferrario_2015B}. In these more massive stars with more massive MS convective cores, the entire mass of the resulting WD was within the MS convective core, and strongly magnetized regions may extend all the way to the WD surface where they can be observed. Thus, many magnetic WDs may be magnetized due to MS convective core dynamos, which could partially explain why magnetic WDs are more massive on average.

The interesting corollary to this discussion is that strong magnetic fields may exist within the interiors of most WDs even though the fields are not visible at the surface. Indeed, based on the MS-WD mass relation, most WDs originate from progenitors of $M \gtrsim 1.5 \, M_\odot$ in which core dynamos operate and produce strong magnetic fields in over 50 \% of RGB stellar cores (Stello et al. 2015). We therefore speculate that the majority of WDs could contain strong ($B \gtrsim 10^6 \, {\rm G}$) magnetic fields which are confined within the stellar interior and not detectable at the surface. These magnetic fields may have very important implications for WD evolution, and for the outcome of WD mergers. 





\cite{Ferrario_2015A}
  
  
  
  
  
  