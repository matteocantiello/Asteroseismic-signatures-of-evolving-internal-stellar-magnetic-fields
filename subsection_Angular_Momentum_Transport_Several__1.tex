\subsection{Angular Momentum Transport}

Several asteroseismic studies \citep{Beck_2012,Mosser_2012,Deheuvels_2014,Deheuvels_2015} have measured the core rotation rates of RGB/clump stars. The relatively slow core rotation rates indicate strong angular momentum transport mechanisms are at work \citep{Cantiello_2014}, coupling the radiative cores with the convective envelope. The strong magnetic fields frequently found in the cores of RGB stars (Stello et al. 2015) may play an important role in this process. Our work suggests that strong magnetic fields restricted to mass coordinates of RGB stars that were convective on the MS. For stars of $M \lesssim 1.5 \, M_\odot$, the strong fields are restricted to the He core, and cannot directly couple the core with the envelope. Since the majority of the sample of \cite{Mosser_2012} has $M \lesssim 1.5 \, M_\odot$, core-dynamo-generated fields cannot solely account for slow core rotation.

More importantly, the sample of stars with measured core rotation rates are mutually exclusive from the sample of stars with strong core magnetic fields. The reason is that mixed dipole modes are used to measure the core rotation rates, but these modes are highly suppressed/absent in stars with magnetic cores. In order for large-scale magnetic fields to account for the measured core rotation rates, they must extend from the inner helium core to the outer radiative core, and they must be weak enough that they do not suppress dipole oscillation modes. Unfortunately, neither the current study nor the measurements of Stello et al. 2015 can provide useful constraints on the existence of such fields. 
