\label{fig:DipoleHist}
Kippenhahn diagrams of the inner regions of intermediate-mass stars, shown from the main sequence to the red clump phases of evolution. Blue shaded regions are convective zones, labeled in the second panel. Pink shaded regions indicate mass coordinates that were within the main sequence convective zone and contain strong core-dynamo-generated fields. The red line is the location of the H-burning shell during post-main-sequence evolution. The dashed green line estimates the central magnetic field strength, $B_{\rm cen}$, calculated via equation \ref{eqn:Bcen}. Dashed vertical lines denote the location of $\nu_{\rm max} = 500 \, \mu{\rm Hz}$ (left line) and $\nu_{\rm max} = 50 \, \mu{\rm Hz}$ (right line). Mixed modes and suppressed modes are generally observable when the star lies between the vertical dashed lines. Magnetic suppression is most likely when the red line lies within a pink shaded region, i.e., it is most likely to be observed for $M \gtrsim 1.5 \, M_\odot$, in accordance with the results of \cite{Stello_2016}. For the same evolutionary stage, the frequency of maximum power shifts to lower values for stars of higher mass, such that $\nu_{\rm max} = 500 \, \mu{\rm Hz}$ corresponds to the main sequence for our most massive models.