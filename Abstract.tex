\begin{itemize}
\item Predictions for $\ell = 2$
\item Predictions for clump 
\item Predictions for buried fields
\end{itemize}



Recent asteroseismic analyses have revealed the presence of strong magnetic fields buried within the cores of many red giant stars. Here, we examine the implications of these results for the coupled evolution of stars and their magnetic fields, and we make predictions for future asteroseismic observations. Strong internal magnetic fields should produce moderate suppression of the visibility of quadrupole oscillation modes, which can be verified with existing data. Long magnetic diffusion times ensure that dynamo-generated fields are confined to mass coordinates within the main sequence convective core, and the sharp increase in dipole mode suppression rates above $1.5 \, M_\odot$ can be explained by the larger convective core masses of more massive stars.
%, allowing them to deposit strong magnetic fields at mass coordinates near the H-burning shell which can suppress oscillation modes during the subsequent red giant phase. 
In clump stars, fields of $\sim 10^5 \, {\rm G}$ can suppress dipole modes, and the visibility of suppressed modes in clump stars is equal to or less than the visibility of suppressed modes in ascending red giants. High dipole mode suppression rates in low-mass ($M \lesssim 2 \, M_\odot$) clump stars would indicate that magnetic fields generated during the main sequence can withstand subsequent convective phases and survive into the compact remnant phase. Finally, we discuss implications for observed magnetic fields in white dwarfs and neutron stars, as well as the effects of magnetic fields in various types of pulsating stars.
  
  
  
  
  