\section{Timescales}
\label{time}
Here we discuss the relevant timescales affecting the evolution of internal magnetic fields and their ability to 
suppress oscillation modes through a magnetic greenhouse effect (Fuller et al. 2015).
We assume that at the end of the main sequence a magnetic field is present below the maximum lagrangian mass occupied by the convective core during core H-burning.

The first important timescale is the Ohmic timescale $t_{\rm Ohm}= H_{\rm P}^2/\eta$, the time it takes a stable magnetic field in a radiative region to diffuse a pressure scale height $H_{\rm P}$. This timescale is usually quite long, due to the typically  small values of the magnetic diffusivity $\eta$ in the stellar plasmas. Figure~\ref{fig:timescales} shows that in the core of a $1.5\mso$ evolving from the main sequence to the RGB,  $t_{\rm Ohm}$ varies between $10^8-10^{12}$ yrs. Therefore we can safely assume that magnetic fields present in the stellar core at the end of the main sequence are frozen in their lagrangian mass coordinate.
Note that  $t_{\rm Ohm}$ does not depend on the amplitude or geometry of the magnetic field, but only on the local value of the magnetic diffusivity. The magnetic diffusivity is the inverse of the electrical conductivity,  which in these stars has to be calculated carefully as certain regions are partially/fully degenerate. Moving from non-degenerate, to partially and fully degenerate regions, we calculate the magnetic diffusivity according to \citec{1968dms..book.....S}, \citec{1987ApJ...313..284W} and \citec{1984MNRAS.209..511N} respectively, applying a smooth interpolation in the transition regions.

The other important timescale is the H-shell burning timescale. As a star moves from the end of its main sequence to the RGB phase, the ashes of H-shell burning increase the size of its He core. We can write the timescale of this process as $t_{\rm Acc} = H_{\rm P} 4\pi r^2 \rho / \dot{m}$, where $r$ is the local radial coordinate and $\dot{m}$ is the He accretion rate. If $t_{\rm Acc} < t_{\rm Ohm}$, then the magnetic field can be buried below the He raining from the H-shell. Figure~\ref{fig:timescales} shows that in a $1.5\mso$ this is always the case. As a consequence if the He core grows substantially during the subgiant / early RGB phases, magnetic fields can be efficiently buried below the H-shell, the location where the waves are most sensitive to the magnetic greenhouse effect (see e.g. the $1.25\mso$ model in                  Fig.~\label{fig:DipoleHist}). Therefore in red giants in the mass range $1.1-1.75\mso$, the expectation is a decrease of magnetic suppression for decreasing initial stellar mass.

\subsection{Magnetic buoyancy}